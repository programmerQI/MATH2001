%%%%%%%%%%%%%%%%%%%%%%%%%%%%%%%%%%%%%%%%%%%% DOCUMENT CLASS %%%%%%%%%%%%%%%%%%%%%%%%%%%

\documentclass[12pt]{amsart}
%\documentclass[draft, 12pt]{amsart}


%%%%%%%%%%%%%%%%%%%%%%%%%%%%%%%%%%%%%%%%%%%%%%%%%% STANDARD PACKAGES %%%%%%%%%%%%%%%%%%%%%%%%%%%%%%%%%%%%%%%%%

\usepackage{amssymb,amsmath,amsthm,amscd,mathrsfs,graphicx, color}
\usepackage[cmtip,all,matrix,arrow,tips,curve]{xy}
\usepackage[notref,notcite]{showkeys}
%\usepackage[colorlinks]{hyperref}
\usepackage{multicol}
\usepackage{hyperref}
\usepackage[usenames,dvipsnames]{xcolor}
\hypersetup{colorlinks=true,citecolor=OliveGreen,linkcolor=BrickRed,urlcolor=BlueViolet}
\usepackage[active]{srcltx}
\usepackage{mathpazo}
\usepackage{setspace}\doublespacing 
%Double space, and make it easier for the grader to grade your homework.
%\usepackage{fullpage} 
%Use wide margins, and make it easier for the grader to grade your homework.


%%%%%%%%%%%%%%%%%%%%%%%%%%%%%%%%%%%%%%%%%%%%%% TIKZ FOR GRAPHING %%%%%%%%%%%%%%%%%%%%%%%%%%%%%%%%%%%%%%%%%%%%%%%%%%%%%%%
\usepackage{tikz,pgfplots}


%%%%%%%%%%%%%%%%%%%%%%%%%%%%%%%%%%%%%%%%%%%%%% 		ANSWER BOXES 		%%%%%%%%%%%%%%%%%%%%%%%%%%%%%%%%%%%%%%%%%%%
 \setlength\fboxsep{.3cm}
\setlength\fboxrule{.05cm}

\newcommand*{\boxedcolor}{red}
\makeatletter
\renewcommand{\boxed}[1]{\textcolor{\boxedcolor}{%
  \fbox{\normalcolor\m@th$\displaystyle#1$}}}
\makeatother

\makeatletter
\newcommand{\boxedred}[1]{\textcolor{red}{%
  \fbox{\normalcolor\m@th$\displaystyle#1$}}}
\makeatother

\makeatletter
\newcommand{\boxedblue}[1]{\textcolor{blue}{%
  \fbox{\normalcolor\m@th$\displaystyle#1$}}}
\makeatother




%%%%%%%%%%%%%%%%%%%%%%%%%%%%%%%%%%%%%%%%%%%%%%%%%%%% THEOREM ENVIRONMENTS %%%%%%%%%%%%%%%%%%%%%%%%%%%%%%%%


\numberwithin{equation}{section}
\newtheorem{teo}{Theorem}[section]
\newtheorem{pro}[teo]{Proposition}
\newtheorem{lem}[teo]{Lemma}
\newtheorem{cor}[teo]{Corollary}
\newtheorem{con}[teo]{Conjecture}
\newtheorem{convention}[teo]{}



\newtheorem{teoalpha}{Theorem}
\renewcommand{\theteoalpha}{\Alph{teoalpha}}
\newtheorem{proalpha}[teoalpha]{Proposition}
\newtheorem{coralpha}[teoalpha]{Corollary}


\theoremstyle{definition}
\newtheorem{dfn}[teo]{Definition}
\newtheorem{exa}[teo]{Example}
\newtheorem{que}[teo]{Question}

\theoremstyle{remark}
\newtheorem{rem}[teo]{Remark}
\newtheorem{nte}[teo]{Note}

%%%%%%%%%%%%%%%%%%%%%%%%%%%%%%%%%%%%%%%%%%%%%%% SOME CONDITIONALS FOR NOTES %%%%%%%%%%%%%%%%%%%%%%%%%%%%%%%%%%%

% Declare a new conditional
\newif\ifnotes
\notestrue 	% Show details
%\notesfalse 	% Exclude details


%%%%%%%%%%%%%%%%%%%%%%%%%%%%%%%%%%%%%%%%%%%%% COMMENTS %%%%%%%%%%%%%%%%%%%%%%%%%%%%

\newcommand{\marg}[1]{\normalsize{{\color{red}\footnote{{\color{blue}#1}}}{\marginpar[\vskip -.3cm {\color{BrickRed}\hfill\thefootnote$\implies$}]{\vskip -.3cm{ \color{BrickRed}$\impliedby$\thefootnote}}}}}

\newcommand{\qc}[1]{\marg{#1}}

%%%%%%%%%%%%%%%%%%%%%%%%%%%%%%%%%%%%%%%%%%%%%%%%%%% BEGIN DOCUMENT %%%%%%%%%%%%%%%%%%%%%%%%%%%%%%%%%%%
 
\begin{document}

\bibliographystyle{amsalpha}


%%%%%%%%%%%%%%%%%%%%%%%%%%%%%%%%%%%%%%%%%%%%%%%% AUTHOR INFO %%%%%%%%%%%%%%%%%%%%%%%%%%%%%%%%%%%% 

\author[QI]{QI WANG}
\address{University of Colorado, Department of Mathematics,  Campus Box 395,
Boulder, CO 80309-0395}
\email{casa@math.colorado.edu}
\date{\today}
%\thanks{I would like to take this opportunity to thank my class for their support.}


%%%%%%%%%%%%%%%%%%%%%%%%%%%%%%%%%%%%%%%%%%%%%%%% TITLE AND ABSTRACT %%%%%%%%%%%%%%%%%%%%%%%%%%%%%%%%%%%%

\title[Homework 8]{Homework 8 \\ \ \\  MATH 2001}

\begin{abstract} 
This is the first homework assignment.  The problems are from Hammack \cite[Ch.~1, \S 1.1]{H13}:
\begin{itemize}

\item \textbf{Chapter 7}  
Exercises:  19, 20, 21

\item \textbf{Chapter 8}  
Exercises:  2, 6, 10, 12, 14, 16, 18

\end{itemize}
\end{abstract}


\maketitle


\tableofcontents

%%%%%%%%%%%%%%%%%%%%%%%%%%%%%%%%%%%%%%%%%%%%%%%% HOMEWORK ASSIGNMENT %%%%%%%%%%%%%%%%%%%%%%%%%%%%%%%%%%%%

%%%%%%%%%%%%%%%%%%%%%%%%%%%%%%%%%%%%%%%%%%%%%%%%%%%%%%%%%%%%%%%%%%%%%%%%%%%%%%%%%%%%%%%%%% CHAPTER 1 %%%%%%%%%%%%%%%%%%%%%%%%%%%%%%%%%%%%%%%%%%%%%%%%%%%%%%%%%%%%%%%%%%%%%%%%%%%%%%%%%%%%%%%%%%%%%%%%%%%


%%%%%%%%%%%%%%%%%%%%%%%%%%%%%%%%%%%%%%%%%%%%%%%% SECTION 1.1 %%%%%%%%%%%%%%%%%%%%%%%%%%%%%%%%%%%%

\section*{Chapter 7}


%%%%%%%%%%%%%%%%%%%%%%%%%%%%%%%%%%%%%%%%%%%%%%%% EXERCISE 2 %%%%%%%%%%%%%%%%%%%%%%%%%%%%%%%%%%%%

\subsection*{Ch.7, Exercise 19}  If $ n \in \mathbb{Z} $, then $ 2^0 + 2^2  + 2^3 + \dots + 2^n = 2^{n+1} - 1 $.


\begin{proof}[Solution to Ch.7, Exercise 19] 
\ \\
\textbf{Proposition} If $ n \in \mathbb{Z} $, then $ 2^0 + 2^2  + 2^3 + \dots + 2^n = 2^{n+1} - 1 $. \\
\textit{Proof}: \\
$ n = 0 $ : \ $ 2^0 = 2^1 - 1 = 1 $ \\
$ n = 1 $ : \ $ 2^0 + 2^1 = 2^2 - 1 = 3 $ \\
$ n = 2 $ : \ $ 2^0 + 2^1 + 2^2 = 2^3 - 1 = 7 $ \\
\textit{Assume}: \\
$ n = k - 1 $ : \ $ 2^0 + 2^1 + 2^2 + \dots + 2^{k - 1} = 2^k - 1 $ \\
\textit{Induction Proof}: \\
$ n = k $ : \ $ 2^0 + 2^1 + 2^2 + \dots + 2^{k - 1} + 2^k = 2^k - 1 + 2^k = 2^{k + 1} - 1 $

\end{proof}


%%%%%%%%%%%%%%%%%%%%%%%%%%%%%%%%%%%%%%%%%%%%%%%% EXERCISE 8 %%%%%%%%%%%%%%%%%%%%%%%%%%%%%%%%%%%%


\subsection*{Ch.7, Exercise 20}  There exists an $ n \in \mathbb{N} $ for which $ 11 | (2^n - 1) $.

\begin{proof}[Solution to Ch.7, Exercise 20]
\ \\
\textbf{Proposition} There exists an $ n \in \mathbb{N} $ for which $ 11 | (2^n - 1) $.\\
\textit{Proof} (direct) \ Because zero divides by eleven equals to zero, we have $ 11 | 0 $. Let $ 2^n - 1 = 0 $, so we get $ n = 0 $. Thus, there exist an $ n $ when $ n = 0 $ for $ 11 | (2^n - 1) $.

\end{proof}


%%%%%%%%%%%%%%%%%%%%%%%%%%%%%%%%%%%%%%%%%%%%%%%% EXERCISE 18 %%%%%%%%%%%%%%%%%%%%%%%%%%%%%%%%%%%%


\subsection*{Ch.8,  Exercise 2}  Prove that $ \{ 6n : n \in \mathbb{Z} \} = \{ 2n : n \in \mathbb{Z} \} \land \{ 3n : n \in \mathbb{Z} \} $.

\begin{proof}[Solution to Ch.8, Exercise 2]
\ \\
\textbf{Proposition}  Prove that $ \{ 6n : n \in \mathbb{Z} \} = \{ 2n : n \in \mathbb{Z} \} \land \{ 3n : n \in \mathbb{Z} \} $. \\
\textit{Proof}  \\
\textbf{Step 1}: Suppose $ a \in \{ 6n : n \in \mathbb{Z} \} $, we have $ a = 6n = 2(3n) $. Thus $ a = 2(b) $ where $ b = 3n \in \mathbb{Z} $, so $ a \in \{ 2n : n \in \mathbb{Z} \} $. We also have $ a = 6n = 3(2n) $, so $ a = 3c $ where $ c = 2n \in \mathbb{Z} $. Thus $ a \in \{ 3n : n \in \mathbb{Z} \} $. Therefor $ \{ 6n : n \in \mathbb{Z} \} \subseteq \{ 2n : n \in \mathbb{Z} \} \land \{ 3n : n \in \mathbb{Z} \} $. \\
\textbf{Step 2}: Suppose $ a \in \{ 2n : n \in \mathbb{Z} \} \land \{ 3n : n \in \mathbb{Z} \} $. Then, we have $ a = 2b $ and $ a = 3c $ where $ b, c \in \mathbb{Z} $. Thus $ 2 | a $ and $ 3 | a $, so $ (2 * 3) | a = 6 | a $.  Therefore $ a = 6d $ for $ d \in \mathbb{Z} $. Thus $ a \in \{ 6n : n \in \mathbb{Z} \} $, so $ \{ 2n : n \in \mathbb{Z} \} \land \{ 3n : n \in \mathbb{Z} \} \subseteq \{ 6n : n \in \mathbb{Z} \} $.

\end{proof}


%%%%%%%%%%%%%%%%%%%%%%%%%%%%%%%%%%%%%%%%%%%%%%%% EXERCISE 30 %%%%%%%%%%%%%%%%%%%%%%%%%%%%%%%%%%%%


\subsection*{Ch.8,  Exercise 6}  Suppose $ x, y \in \mathbb{R} $. Then $ x^3 + x^2y = y^2 + xy $ if and only if $ y = x^2 $ or $ y = -x $.

\begin{proof}[Solution to Ch.1, \S 1.1,  Exercise 30] 
\ \\
\textbf{Proposition} Suppose $ x, y \in \mathbb{R} $. Then $ x^3 + x^2y = y^2 + xy $ if and only if $ y = x^2 $ or $ y = -x $.\\
\textit{Proof} Rearranging the equation $ x^3 + x^2y = y^2 + xy $, we get $ x^3 - xy = y^2 - x^2 y $. Thus $ x ( x^2 - y ) = y ( y - x^2 ) $, so we have $ x ( x^2 - y) = - y ( x^2 - y) $. Therefore we solve the equation get either $ y = x^2 $ or $ y = -x $.
\end{proof}



%%%%%%%%%%%%%%%%%%%%%%%%%%%%%%%%%%%%%%%%%%%%%%%% EXERCISE 38 %%%%%%%%%%%%%%%%%%%%%%%%%%%%%%%%%%%%


\subsection*{Ch.8, Exercise 10} If $ a \in \mathbb{Z} $, then $ a^3 \equiv a (mod 3) $.

\begin{proof}[Solution to Ch.8, Exercise 12]
\ \\
\textbf{Proposition} If $ a \in \mathbb{Z} $, then $ a^3 \equiv a (mod \  3) $.\\
\textit{Proof (proof by case)} \\
\textbf{case 1:} \ Suppose $ a \equiv 1 (mod \ 3) $. We get $ a = 3n + 1, n \in \mathbb{Z} $. Thus $ a^3 = 27n^3 + 27n^2 + 9b + 1 = 3 (9b^3 + 9b^2 + 3b) + 1 $ where $ 9b^3 + 9b^2 + 3b \in \mathbb{Z} $, so $ a^3 \equiv 1 (mod \ 3) $. Therefor $ a^3 \equiv a (mod \ 3) $.\\
\textbf{case 2:} \ Suppose $ a \equiv 2 (mod \ 3) $. We get $ a = 3n + 2, n \in \mathbb{Z} $. Thus $ a^3 = 27n^3 + 54n^2 + 36b + 8 = 3 (9b^3 + 18b^2 + 12b + 2) + 2 $ where $ 9b^3 + 18b^2 + 12b + 2 \in \mathbb{Z} $, so $ a^3 \equiv 2 (mod \ 3) $. Therefor $ a^3 \equiv a (mod \ 3) $.\\
\textbf{case 3:} \ Suppose $ a \equiv 0 (mod \ 3) $. We get $ a = 3n, n \in \mathbb{Z} $. Thus $ a^3 = 27n^3 = 3 (9b^3) $ where $ 9b^3 \in \mathbb{Z} $, so $ a^3 \equiv 0 (mod \ 3) $. Therefor $ a^3 \equiv a (mod \ 3) $.\\
For each case $ a^3 \equiv a (mod \ 3) $.
\end{proof}


%%%%%%%%%%%%%%%%%%%%%%%%%%%%%%%%%%%%%%%%%%%%%%%% EXERCISE 40 %%%%%%%%%%%%%%%%%%%%%%%%%%%%%%%%%%%%


\subsection*{Ch.8, Exercise 12}  There exist a positive real number $ x $ for which $ x^2 < \sqrt{x} $.

\begin{proof}[Solution to Ch.8, Exercise 12]
\ \\
\textbf{Proposition} There exist a positive real number $ x $ for which $ x^2 < \sqrt{x} $.
\textit{Proof} Assume $ x = 0.25 $, then $ x^2 = 0.0625 $ and $ \sqrt{x} = 0.5 $.

\end{proof}


%%%%%%%%%%%%%%%%%%%%%%%%%%%%%%%%%%%%%%%%%%%%%%%% EXERCISE 40 %%%%%%%%%%%%%%%%%%%%%%%%%%%%%%%%%%%%


\subsection*{Ch.8, Exercise 14}  Suppose $ a \in \mathbb{Z} $. Then $ a^2 | a $ if and only if $ a \in \{ -1, 0, 1 \} $.

\begin{proof}[Solution to Ch.8, Exercise 14]
\ \\
\textbf{Proposition} Suppose $ a \in \mathbb{Z} $. Then $ a^2 | a $ if and only if $ a \in \{ -1, 0, 1 \} $. \\

\textit{Proof} Assume $ a^2 | a $, we get $ a / a^2 = n $, where $ n \in \mathbb{Z} $. By solving the equation, we get $ a = 1, a = -1 $ or $ a = 0 $.

\end{proof}


%%%%%%%%%%%%%%%%%%%%%%%%%%%%%%%%%%%%%%%%%%%%%%%% EXERCISE 40 %%%%%%%%%%%%%%%%%%%%%%%%%%%%%%%%%%%%


\subsection*{Ch.8, Exercise 16}  Suppose $ a, b \in \mathbb{Z} $. If $ ab $ is odd, then $ a^2 + b^2 $ is even.

\begin{proof}[Solution to Ch.8, Exercise 16]
\ \\
\textbf{Proposition} Suppose $ a, b \in \mathbb{Z} $. If $ ab $ is odd, then $ a^2 + b^2 $ is even.\\
\textit{Proof} Suppose $ a $ and $ b $ are odd. We have $ a = 2n + 1 $ and $ b = 2m + 1 $. Thus $ a^2 + b^2 = 4n^2 + 4n + 1 + 4m^2 + 4m + 1 = 2(2n^2 + 2n + 2m^2 + 2m + 1) $ where $ 2n^2 + 2n + 2m^2 + 2m + 1 \in \mathbb{Z} $. Therefore $ a^2 + b^2 $ is even.

\end{proof}


%%%%%%%%%%%%%%%%%%%%%%%%%%%%%%%%%%%%%%%%%%%%%%%% EXERCISE 40 %%%%%%%%%%%%%%%%%%%%%%%%%%%%%%%%%%%%


\subsection*{Ch.8, Exercise 18}  There is a set $ X $ for which $ \mathbb{N} \in X $ and $ \mathbb{N} \subseteq X $.

\begin{proof}[Solution to Ch.8, Exercise 18]
\ \\
\textbf{Proposition} There is a set $ X $ for which $ \mathbb{N} \in X $ and $ \mathbb{N} \subseteq X $.\\
\textit{Proof} There is $ \wp \mathbb{N}$ which $ \mathbb{N} \in \wp \mathbb{N} $ and $ \mathbb{N} \subseteq \wp  \mathbb{N} $.

\end{proof}

\ifnotes


\else
	This is not the full version.  This can be useful if there is scratch work you want to keep for yourself, but you do not want other people to see. 
\fi




\bibliography{templateHW}
\end{document}
