%%%%%%%%%%%%%%%%%%%%%%%%%%%%%%%%%%%%%%%%%%%% DOCUMENT CLASS %%%%%%%%%%%%%%%%%%%%%%%%%%%

\documentclass[12pt]{amsart}
%\documentclass[draft, 12pt]{amsart}


%%%%%%%%%%%%%%%%%%%%%%%%%%%%%%%%%%%%%%%%%%%%%%%%%% STANDARD PACKAGES %%%%%%%%%%%%%%%%%%%%%%%%%%%%%%%%%%%%%%%%%

\usepackage{amssymb,amsmath,amsthm,amscd,mathrsfs,graphicx, color}
\usepackage[cmtip,all,matrix,arrow,tips,curve]{xy}
\usepackage[notref,notcite]{showkeys}
%\usepackage[colorlinks]{hyperref}
\usepackage{multicol}
\usepackage{hyperref}
\usepackage[usenames,dvipsnames]{xcolor}
\hypersetup{colorlinks=true,citecolor=OliveGreen,linkcolor=BrickRed,urlcolor=BlueViolet}
\usepackage[active]{srcltx}
\usepackage{mathpazo}
\usepackage{setspace}\doublespacing 
%Double space, and make it easier for the grader to grade your homework.
%\usepackage{fullpage} 
%Use wide margins, and make it easier for the grader to grade your homework.


%%%%%%%%%%%%%%%%%%%%%%%%%%%%%%%%%%%%%%%%%%%%%% TIKZ FOR GRAPHING %%%%%%%%%%%%%%%%%%%%%%%%%%%%%%%%%%%%%%%%%%%%%%%%%%%%%%%
\usepackage{tikz,pgfplots}


%%%%%%%%%%%%%%%%%%%%%%%%%%%%%%%%%%%%%%%%%%%%%% 		ANSWER BOXES 		%%%%%%%%%%%%%%%%%%%%%%%%%%%%%%%%%%%%%%%%%%%
 \setlength\fboxsep{.3cm}
\setlength\fboxrule{.05cm}

\newcommand*{\boxedcolor}{red}
\makeatletter
\renewcommand{\boxed}[1]{\textcolor{\boxedcolor}{%
  \fbox{\normalcolor\m@th$\displaystyle#1$}}}
\makeatother

\makeatletter
\newcommand{\boxedred}[1]{\textcolor{red}{%
  \fbox{\normalcolor\m@th$\displaystyle#1$}}}
\makeatother

\makeatletter
\newcommand{\boxedblue}[1]{\textcolor{blue}{%
  \fbox{\normalcolor\m@th$\displaystyle#1$}}}
\makeatother




%%%%%%%%%%%%%%%%%%%%%%%%%%%%%%%%%%%%%%%%%%%%%%%%%%%% THEOREM ENVIRONMENTS %%%%%%%%%%%%%%%%%%%%%%%%%%%%%%%%


\numberwithin{equation}{section}
\newtheorem{teo}{Theorem}[section]
\newtheorem{pro}[teo]{Proposition}
\newtheorem{lem}[teo]{Lemma}
\newtheorem{cor}[teo]{Corollary}
\newtheorem{con}[teo]{Conjecture}
\newtheorem{convention}[teo]{}



\newtheorem{teoalpha}{Theorem}
\renewcommand{\theteoalpha}{\Alph{teoalpha}}
\newtheorem{proalpha}[teoalpha]{Proposition}
\newtheorem{coralpha}[teoalpha]{Corollary}


\theoremstyle{definition}
\newtheorem{dfn}[teo]{Definition}
\newtheorem{exa}[teo]{Example}
\newtheorem{que}[teo]{Question}

\theoremstyle{remark}
\newtheorem{rem}[teo]{Remark}
\newtheorem{nte}[teo]{Note}

%%%%%%%%%%%%%%%%%%%%%%%%%%%%%%%%%%%%%%%%%%%%%%% SOME CONDITIONALS FOR NOTES %%%%%%%%%%%%%%%%%%%%%%%%%%%%%%%%%%%

% Declare a new conditional
\newif\ifnotes
\notestrue 	% Show details
%\notesfalse 	% Exclude details


%%%%%%%%%%%%%%%%%%%%%%%%%%%%%%%%%%%%%%%%%%%%% COMMENTS %%%%%%%%%%%%%%%%%%%%%%%%%%%%

\newcommand{\marg}[1]{\normalsize{{\color{red}\footnote{{\color{blue}#1}}}{\marginpar[\vskip -.3cm {\color{BrickRed}\hfill\thefootnote$\implies$}]{\vskip -.3cm{ \color{BrickRed}$\impliedby$\thefootnote}}}}}

\newcommand{\qc}[1]{\marg{#1}}

%%%%%%%%%%%%%%%%%%%%%%%%%%%%%%%%%%%%%%%%%%%%%%%%%%% BEGIN DOCUMENT %%%%%%%%%%%%%%%%%%%%%%%%%%%%%%%%%%%
 
\begin{document}

\bibliographystyle{amsalpha}


%%%%%%%%%%%%%%%%%%%%%%%%%%%%%%%%%%%%%%%%%%%%%%%% AUTHOR INFO %%%%%%%%%%%%%%%%%%%%%%%%%%%%%%%%%%%% 

\author[Casalaina]{Sebastian Casalaina}
\address{University of Colorado, Department of Mathematics,  Campus Box 395,
Boulder, CO 80309-0395}
\email{casa@math.colorado.edu}
\date{\today}
%\thanks{I would like to take this opportunity to thank my class for their support.}


%%%%%%%%%%%%%%%%%%%%%%%%%%%%%%%%%%%%%%%%%%%%%%%% TITLE AND ABSTRACT %%%%%%%%%%%%%%%%%%%%%%%%%%%%%%%%%%%%

\title[Homework 11]{Homework 11 \\ \ \\  MATH 2001}

\begin{abstract} 
This is the first homework assignment.  The problems are from Hammack \cite[Ch.~1, \S 1.1]{H13}:
\begin{itemize}

\item \textbf{Chapter 11}  
\textbf{Section 11.3}, Exercises:  2, 4.

\end{itemize}
\end{abstract}


\maketitle


\tableofcontents

%%%%%%%%%%%%%%%%%%%%%%%%%%%%%%%%%%%%%%%%%%%%%%%% HOMEWORK ASSIGNMENT %%%%%%%%%%%%%%%%%%%%%%%%%%%%%%%%%%%%

%%%%%%%%%%%%%%%%%%%%%%%%%%%%%%%%%%%%%%%%%%%%%%%%%%%%%%%%%%%%%%%%%%%%%%%%%%%%%%%%%%%%%%%%%% CHAPTER 1 %%%%%%%%%%%%%%%%%%%%%%%%%%%%%%%%%%%%%%%%%%%%%%%%%%%%%%%%%%%%%%%%%%%%%%%%%%%%%%%%%%%%%%%%%%%%%%%%%%%


%%%%%%%%%%%%%%%%%%%%%%%%%%%%%%%%%%%%%%%%%%%%%%%% SECTION 1.1 %%%%%%%%%%%%%%%%%%%%%%%%%%%%%%%%%%%%

\section*{Chapter 11 Section 11.3}


%%%%%%%%%%%%%%%%%%%%%%%%%%%%%%%%%%%%%%%%%%%%%%%% EXERCISE 2 %%%%%%%%%%%%%%%%%%%%%%%%%%%%%%%%%%%%

\subsection*{Ch.11, \S 11.3,  Exercise 2}  List all the partitions of the set  $ A = /{a, b, c /} $.

\begin{proof}[Solution to Ch.1, \S 1.1,  Exercise 2] 
$$ \{ \{ a \}, \{ b \}, \{ c \} \} $$
$$ \{ \{ a \}, \{ a, b \} \} $$
$$ \{ \{ a, b \}, \{ c \} \} $$
$$ \{ \{ a, b, c \} \} $$
\end{proof}





\ifnotes


\else
	This is not the full version.  This can be useful if there is scratch work you want to keep for yourself, but you do not want other people to see. 
\fi




\bibliography{templateHW}
\end{document}
