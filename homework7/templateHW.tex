%%%%%%%%%%%%%%%%%%%%%%%%%%%%%%%%%%%%%%%%%%%% DOCUMENT CLASS %%%%%%%%%%%%%%%%%%%%%%%%%%%

\documentclass[12pt]{amsart}
%\documentclass[draft, 12pt]{amsart}


%%%%%%%%%%%%%%%%%%%%%%%%%%%%%%%%%%%%%%%%%%%%%%%%%% STANDARD PACKAGES %%%%%%%%%%%%%%%%%%%%%%%%%%%%%%%%%%%%%%%%%

\usepackage{amssymb,amsmath,amsthm,amscd,mathrsfs,graphicx, color}
\usepackage[cmtip,all,matrix,arrow,tips,curve]{xy}
\usepackage[notref,notcite]{showkeys}
%\usepackage[colorlinks]{hyperref}
\usepackage{multicol}
\usepackage{hyperref}
\usepackage[usenames,dvipsnames]{xcolor}
\hypersetup{colorlinks=true,citecolor=OliveGreen,linkcolor=BrickRed,urlcolor=BlueViolet}
\usepackage[active]{srcltx}
\usepackage{mathpazo}
\usepackage{setspace}\doublespacing 
%Double space, and make it easier for the grader to grade your homework.
%\usepackage{fullpage} 
%Use wide margins, and make it easier for the grader to grade your homework.


%%%%%%%%%%%%%%%%%%%%%%%%%%%%%%%%%%%%%%%%%%%%%% TIKZ FOR GRAPHING %%%%%%%%%%%%%%%%%%%%%%%%%%%%%%%%%%%%%%%%%%%%%%%%%%%%%%%
\usepackage{tikz,pgfplots}


%%%%%%%%%%%%%%%%%%%%%%%%%%%%%%%%%%%%%%%%%%%%%% 		ANSWER BOXES 		%%%%%%%%%%%%%%%%%%%%%%%%%%%%%%%%%%%%%%%%%%%
 \setlength\fboxsep{.3cm}
\setlength\fboxrule{.05cm}

\newcommand*{\boxedcolor}{red}
\makeatletter
\renewcommand{\boxed}[1]{\textcolor{\boxedcolor}{%
  \fbox{\normalcolor\m@th$\displaystyle#1$}}}
\makeatother

\makeatletter
\newcommand{\boxedred}[1]{\textcolor{red}{%
  \fbox{\normalcolor\m@th$\displaystyle#1$}}}
\makeatother

\makeatletter
\newcommand{\boxedblue}[1]{\textcolor{blue}{%
  \fbox{\normalcolor\m@th$\displaystyle#1$}}}
\makeatother




%%%%%%%%%%%%%%%%%%%%%%%%%%%%%%%%%%%%%%%%%%%%%%%%%%%% THEOREM ENVIRONMENTS %%%%%%%%%%%%%%%%%%%%%%%%%%%%%%%%


\numberwithin{equation}{section}
\newtheorem{teo}{Theorem}[section]
\newtheorem{pro}[teo]{Proposition}
\newtheorem{lem}[teo]{Lemma}
\newtheorem{cor}[teo]{Corollary}
\newtheorem{con}[teo]{Conjecture}
\newtheorem{convention}[teo]{}



\newtheorem{teoalpha}{Theorem}
\renewcommand{\theteoalpha}{\Alph{teoalpha}}
\newtheorem{proalpha}[teoalpha]{Proposition}
\newtheorem{coralpha}[teoalpha]{Corollary}


\theoremstyle{definition}
\newtheorem{dfn}[teo]{Definition}
\newtheorem{exa}[teo]{Example}
\newtheorem{que}[teo]{Question}

\theoremstyle{remark}
\newtheorem{rem}[teo]{Remark}
\newtheorem{nte}[teo]{Note}

%%%%%%%%%%%%%%%%%%%%%%%%%%%%%%%%%%%%%%%%%%%%%%% SOME CONDITIONALS FOR NOTES %%%%%%%%%%%%%%%%%%%%%%%%%%%%%%%%%%%

% Declare a new conditional
\newif\ifnotes
\notestrue 	% Show details
%\notesfalse 	% Exclude details


%%%%%%%%%%%%%%%%%%%%%%%%%%%%%%%%%%%%%%%%%%%%% COMMENTS %%%%%%%%%%%%%%%%%%%%%%%%%%%%

\newcommand{\marg}[1]{\normalsize{{\color{red}\footnote{{\color{blue}#1}}}{\marginpar[\vskip -.3cm {\color{BrickRed}\hfill\thefootnote$\implies$}]{\vskip -.3cm{ \color{BrickRed}$\impliedby$\thefootnote}}}}}

\newcommand{\qc}[1]{\marg{#1}}

%%%%%%%%%%%%%%%%%%%%%%%%%%%%%%%%%%%%%%%%%%%%%%%%%%% BEGIN DOCUMENT %%%%%%%%%%%%%%%%%%%%%%%%%%%%%%%%%%%
 
\begin{document}

\bibliographystyle{amsalpha}


%%%%%%%%%%%%%%%%%%%%%%%%%%%%%%%%%%%%%%%%%%%%%%%% AUTHOR INFO %%%%%%%%%%%%%%%%%%%%%%%%%%%%%%%%%%%% 

\author[Casalaina]{Sebastian Casalaina}
\address{University of Colorado, Department of Mathematics,  Campus Box 395,
Boulder, CO 80309-0395}
\email{casa@math.colorado.edu}
\date{\today}
%\thanks{I would like to take this opportunity to thank my class for their support.}


%%%%%%%%%%%%%%%%%%%%%%%%%%%%%%%%%%%%%%%%%%%%%%%% TITLE AND ABSTRACT %%%%%%%%%%%%%%%%%%%%%%%%%%%%%%%%%%%%

\title[Homework 7]{Homework 7 \\ \ \\  MATH 2001}

\begin{abstract} 
This is the first homework assignment.  The problems are from Hammack \cite[Ch.5]{H13}:
\begin{itemize}

\item \textbf{Chapter 5}  
\textbf{}, Exercises:  1, 2, 3, 4, 5, 16, 17, 18, 19, 20

\end{itemize}
\end{abstract}


\maketitle


\tableofcontents

%%%%%%%%%%%%%%%%%%%%%%%%%%%%%%%%%%%%%%%%%%%%%%%% HOMEWORK ASSIGNMENT %%%%%%%%%%%%%%%%%%%%%%%%%%%%%%%%%%%%

%%%%%%%%%%%%%%%%%%%%%%%%%%%%%%%%%%%%%%%%%%%%%%%%%%%%%%%%%%%%%%%%%%%%%%%%%%%%%%%%%%%%%%%%%% CHAPTER 1 %%%%%%%%%%%%%%%%%%%%%%%%%%%%%%%%%%%%%%%%%%%%%%%%%%%%%%%%%%%%%%%%%%%%%%%%%%%%%%%%%%%%%%%%%%%%%%%%%%%


%%%%%%%%%%%%%%%%%%%%%%%%%%%%%%%%%%%%%%%%%%%%%%%% SECTION 1.1 %%%%%%%%%%%%%%%%%%%%%%%%%%%%%%%%%%%%

\section*{Chapter 5}


%%%%%%%%%%%%%%%%%%%%%%%%%%%%%%%%%%%%%%%%%%%%%%%% EXERCISE 2 %%%%%%%%%%%%%%%%%%%%%%%%%%%%%%%%%%%%

\subsection*{Ch.5,  Exercise 1}  Suppose $ n \in \mathbb{Z} $. If $ n^2 $ is even, then n is even.


\begin{proof}[Solution to Ch.5,  Exercise 1]

\end{proof}


%%%%%%%%%%%%%%%%%%%%%%%%%%%%%%%%%%%%%%%%%%%%%%%% EXERCISE 8 %%%%%%%%%%%%%%%%%%%%%%%%%%%%%%%%%%%%


\subsection*{Ch.5,  Exercise 2}  Suppose $ n \in \mathbb{Z} $. If $ n^2 $ is odd, then n is odd.  


\begin{proof}[Solution to Ch.5,  Exercise 2]

\end{proof}


%%%%%%%%%%%%%%%%%%%%%%%%%%%%%%%%%%%%%%%%%%%%%%%% EXERCISE 18 %%%%%%%%%%%%%%%%%%%%%%%%%%%%%%%%%%%%


\subsection*{Ch.5,  Exercise 3}  Suppose $ a,b \in \mathbb{Z} $. If $ a^2(b^2-2b) $ is odd, then $ a $ and $ b $ are odd.



\begin{proof}[Solution to Ch.5,  Exercise 3]

\end{proof}


%%%%%%%%%%%%%%%%%%%%%%%%%%%%%%%%%%%%%%%%%%%%%%%% EXERCISE 30 %%%%%%%%%%%%%%%%%%%%%%%%%%%%%%%%%%%%


\subsection*{Ch.5,  Exercise 4}  Suppose $ a,b,c \in \mathbb{Z} $. If $ a $ does not divide $ bc $, then $ a $ does not divide $ b $.



\begin{proof}[Solution to Ch.5,  Exercise 4]

\end{proof}



%%%%%%%%%%%%%%%%%%%%%%%%%%%%%%%%%%%%%%%%%%%%%%%% EXERCISE 38 %%%%%%%%%%%%%%%%%%%%%%%%%%%%%%%%%%%%


\subsection*{Ch.5,  Exercise 5}  Suppose $ x \in \mathbb{R} $. If $ x^2 + 5x < 0 $ then $ x < 0 $.



\begin{proof}[Solution to Ch.5,  Exercise 5]

\end{proof}


%%%%%%%%%%%%%%%%%%%%%%%%%%%%%%%%%%%%%%%%%%%%%%%% EXERCISE 40 %%%%%%%%%%%%%%%%%%%%%%%%%%%%%%%%%%%%


\subsection*{Ch.5,  Exercise 16}  Suppose $ x,y \in \mathbb{Z} $. If $ x + y $ is even, then $ x $ and $ y $ have the same parity.


\begin{proof}[Solution to Ch.5,  Exercise 16]

\end{proof}


%%%%%%%%%%%%%%%%%%%%%%%%%%%%%%%%%%%%%%%%%%%%%%%% EXERCISE 40 %%%%%%%%%%%%%%%%%%%%%%%%%%%%%%%%%%%%


\subsection*{Ch.5,  Exercise 17}  If $ n $ is odd, then $ 8 | (n^2 - 1) $.


\begin{proof}[Solution to Ch.5,  Exercise 17]

\end{proof}

%%%%%%%%%%%%%%%%%%%%%%%%%%%%%%%%%%%%%%%%%%%%%%%% EXERCISE 40 %%%%%%%%%%%%%%%%%%%%%%%%%%%%%%%%%%%%


\subsection*{Ch.5,  Exercise 18}  If $ a, b \in \mathbb{Z} $, then $ (a + b)^3 \equiv a^3 + b^3 (mod 3) $.



\begin{proof}[Solution to Ch.5,  Exercise 18]

\end{proof}

%%%%%%%%%%%%%%%%%%%%%%%%%%%%%%%%%%%%%%%%%%%%%%%% EXERCISE 40 %%%%%%%%%%%%%%%%%%%%%%%%%%%%%%%%%%%%


\subsection*{Ch.5,  Exercise 19}  Let $ a,b,c \in \mathbb{Z} $ and $ n \in \mathbb{N} $. If $ a \equiv b(mod n) $ and $ a \equiv c(mod n) $, then $ c \equiv b(mod n) $.



\begin{proof}[Solution to Ch.5,  Exercise 19]

\end{proof}

%%%%%%%%%%%%%%%%%%%%%%%%%%%%%%%%%%%%%%%%%%%%%%%% EXERCISE 40 %%%%%%%%%%%%%%%%%%%%%%%%%%%%%%%%%%%%


\subsection*{Ch.5,  Exercise 20}  If $ a \in \mathbb{Z} $ and $ a \equiv 1(mod 5) $, then $ a^2 \equiv 1(mod 5) $.



\begin{proof}[Solution to Ch.5,  Exercise 20]

\end{proof}


\ifnotes


\else
	This is not the full version.  This can be useful if there is scratch work you want to keep for yourself, but you do not want other people to see. 
\fi




\bibliography{templateHW}
\end{document}
