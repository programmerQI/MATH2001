\documentclass[addpoints]{exam}[14pt]
\usepackage{fullpage}
\usepackage{amssymb,color, amsmath,amsthm,amscd,mathrsfs,graphicx,mathpazo, enumerate,multicol,tabulary}
\usepackage[cmtip,all]{xy}
\usepackage[none]{hyphenat}
\usepackage[dvipsnames]{xcolor}
\usepackage{setspace}




%%%%%%%%%%%%%%%%%%%%%%%%%%%%%%%%%%%%%%%%%%%%%%%%%%%%%%%%%%%%%%%%%%%%%%%%%%%%%%%% CLASS/EXAM NUMBER %%%%%%%%%%%%%%%%%%%%%%%%%%%%%%%%%%%%%%%%%%%%%%%%%%%%%%%%%%%%%%%%%%%%%%%%%%%%%%%%

% Keep these all in one place to prevent typos
\newcommand{\CLASS}{Intro to Discrete Math}
\newcommand{\NUMBER}{MATH 2001}
\newcommand{\SEMESTER}{Spring 2020}
\newcommand{\DATE}{Saturday May 2, 2020}
\newcommand{\EXAM}{Final Exam}
\newcommand{\DURATION}{Take Home Exam}

 %%%%%%%%%%%%%%%%%%%%%%%%%%%%%%%%%%%%%%%%%%%%%%%%%%%%%%%%%%%%%%%%%%%%%%%%%%%%%%%% FORMATTING  %%%%%%%%%%%%%%%%%%%%%%%%%%%%%%%%%%%%%%%%%%%%%%%%%%%%%%%%%%%%%%%%%%%%%%%%%%%%%%%%

\pagestyle{headandfoot}
\cfoot{Page \thepage\mbox{ of }\numpages} % page numbering
%\header{\CLASS}{\EXAM}{\DATE}
%\firstpageheader{\CLASS}{\EXAM}{\DATE}
\setlength{\textheight}{9.25in} \setlength{\voffset}{-0.2in}
\renewcommand{\thequestion}{\bfseries\arabic{question}} %BOLD QUESTION NUMBERS

%%%%%%%%%%%%%%%%%%%%%%%%%%%%%%%%%%%%%%%%%%%%%% TIKZ STUFF FOR GRAPH TO RENDER %%%%%%%%%%%%%%%%%%%%%%%%%%%%%%%%%%%%%%%%%%%%%%%%%%%%%%%
\usepackage{tikz,pgfplots}
\pgfplotsset{samples=500} % Quality of render
\usepgfplotslibrary{fillbetween}
\usepackage{sidecap} % Side caption in figures

%%%%%%%%%%%%%%%%%%%%%%%%%%%%%%%%%%%%%%%%%%%%% SCORE BOXES ON TEST PAGES  %%%%%%%%%%%%%%%%%%%%%%%%%%%%%%%%%%%%%%%%%%%%%%%%%%%%
 
\newcounter{prb_num}
\newcounter{prb_num_a}
\newcommand{\prbbox}[1]
{
\stepcounter{prb_num}
\setcounter{prb_num_a}{0}
\vfill 
\begin{flushright}
\begin{tabular}{|l|} \hline 
\arabic{prb_num} \hskip .2in \ \\  \\ \hline
 #1\textnormal{  points}  \\
 \hline   \end{tabular} 
\end{flushright}
\newpage 
}

%%%%%%%%%%%%%%%%%%%%%%%%%%%%%%%%%%%%%%%%%%%%% T/F BOXES %%%%%%%%%%%%%%%%%%%%%%%%%%%%%%%%%%%%%%%%%%%%%

\newcommand{\prbboxtf}
{
\vskip .01cm
\noindent  \begin{tabular}{r|}
\textbf{T} \ \hskip .5 cm \  \textbf{F}\\
\hline
\end{tabular} 
%\vskip .1 cm 
\hskip .5 cm 
}

%%%%%%%%%%%%%%%%%%%%%%%%%%%%%%%%%%%%%%%%%%%%%% 		EQUATION BOXES 		%%%%%%%%%%%%%%%%%%%%%%%%%%%%%%%%%%%%%%%%%%%

%%%%% SIMPLE SOLUTION %%%%%%%%%%%
 \setlength\fboxsep{.3cm}
\setlength\fboxrule{.05cm}

\newcommand*{\boxedcolor}{red}
\makeatletter
\renewcommand{\boxed}[1]{\textcolor{\boxedcolor}{%
  \fbox{\normalcolor\m@th$\displaystyle#1$}}}
\makeatother

\makeatletter
\newcommand{\boxedred}[1]{\textcolor{red}{%
  \fbox{\normalcolor\m@th$\displaystyle#1$}}}
\makeatother

\makeatletter
\newcommand{\boxedblue}[1]{\textcolor{blue}{%
  \fbox{\normalcolor\m@th$\displaystyle#1$}}}
\makeatother


%%%%%%%%%%%%%%%%%%%%%%%%%%%%%%%%%%%%%%%%%%%%%%%%%%% BEGIN DOCUMENT %%%%%%%%%%%%%%%%%%%%%%%%%%%%%%%%%%%%%%%%%%%%%%%%%%%%%%%%%%%%%%%%%%%%%%%%%%%%%%%%%%%%%%%%%%%%%%

\begin{document}

%%%%%%%%%%%%%%%%%%%%%%%%%%%%%%%%%%%%%%%%%%%%%%%%%%% COVER PAGE  %%%%%%%%%%%%%%%%%%%%%%%%%%%%%%%%%%%%%%%%%%%%%%%%%%%%%%%%%%%%%%%%%%%%%%%%%%%%%%%%%%%%%%%%%%%%%%

\begin{coverpages}
\centerline{\bf\Huge  \EXAM}
\vskip 1cm 
\centerline{\bf\Large \CLASS}
\vskip .2 cm 
\centerline{\bf\Large \NUMBER}
\vskip .2 cm 
\centerline{\bf\Large \SEMESTER}
\vskip .2 cm 
\centerline{\large \DATE}

\vfill
\vskip0.2in
\noindent{\large \textsc{PRINT your name:}} 
\hrulefill
\vfill
\gradetable[v][questions]

\vfill 
\begin{itemize}
\item For the exam, you may use only the textbook, your lecture notes, your homework, and your previous exams, from this course.  

\item You may not use any other resources whatsoever.

\item You may not discuss the exam with anyone, in any way, under any circumstances.

\item \textbf{You must explain your answers, and you will be graded on the clarity of your solutions.}

\item You must write your solutions in \LaTeX, and you must upload your .pdf \emph{and} .tex files to canvas.



\item The exam is due at 11:59 PM Saturday May 2, 2020. 
%, and the exam is \numpoints {} points.
\end{itemize}
\vfill
\end{coverpages}
\vfill\eject













\newpage 

%%%%%%%%%%%%%%%%%%%%%%%%%%%%%%%%%%%%%%%%%%%%%%%%%%% BEGIN EXAM %%%%%%%%%%%%%%%%%%%%%%%%%%%%%%%%%%%%%%%%%%%%%%%%%%%%%%%%%%%%%%%%%%%%%%%%%%%%%%%%%%%%%%%%%%%%%%

\begin{questions}
 
 
 
 
 
 
 
 
 
 
 %%%%%%%%%%%%%%%%%%%%%%%%%%%%%%%%%%%%%%%%%%%%%%%%%%%%%%%%%%%%%%%%%%%%%%%%%%%%%%%%%
 
 
 
 \question[10] 
 
\textbf{True or false.}  

\begin{quote}
\emph{Let $f:A\to B$ and $g:B\to C$ be maps of sets.  If $g\circ f$ is injective, then  $g$  is injective.
}
\end{quote}
\vskip .1 cm 
\textbf{  If true, give a proof.  If false, provide a counter example.}

\begin{proof} \ \\
\doublespacing
False.\\
\textit{Example:}

\usetikzlibrary{positioning,shapes,fit,arrows}
\definecolor{myblue}{RGB}{56,94,141}
\begin{tikzpicture}[line width=1pt,>=latex]
\sffamily

\node (aux1) {};
\node[below=of aux1] (a1) {a};
\node[below=0.5cm of a1] (a2) {b};
\node[below=0.5cm of a2] (aux2) {};

\node[right=4cm of aux1] (aux3) {};
\node[below=0.5cm of aux3] (b1) {1};
\node[below=0.5cm of b1] (b2) {2};
\node[below=0.5cm of b2] (b3) {3};
\node[below=0.5cm of b3] (aux4) {};

\node[right=8cm of aux1] (aux5) {};
\node[below= 0.5cm of aux5] (c1) {q};
\node[below= 0.5cm of c1] (c2) {w};
\node[below= 0.5cm of c2] (aux6) {};

\node[shape=ellipse,draw=myblue,minimum size=3cm,fit={(aux1) (aux2)}] {};
\node[shape=ellipse,draw=myblue,minimum size=3cm,fit={(aux3) (aux4)}] {};
\node[shape=ellipse,draw=myblue,minimum size=3cm,fit={(aux5) (aux6)}] {};

\node[below=1.5cm of aux2,font=\color{myblue}\Large\bfseries] {A};
\node[below=1.5cm of aux4,font=\color{myblue}\Large\bfseries] {B};
\node[below=1.5cm of aux6,font=\color{myblue}\Large\bfseries] {C};

\draw[->,myblue] (a1) -- (b1.170);
%\draw[->,myblue] (a1) -- (b2.170);
\draw[->,myblue] (a2) -- (b2.170);
\draw[->,myblue] (b1) -- (c1.170);
\draw[->,myblue] (b2) -- (c2.170);
\draw[->,myblue] (b3) -- (c2.170);
\end{tikzpicture}

$ A = \{a, b\}, B = \{1, 2, 3\}, C = \{q, w\} $\\
The map $ g \circ f $ is injective, but $ g $ is not.


\end{proof}




 \prbbox{10}



%%%%%%%%%%%%%%%%%%%%%%%%%%%%%%%%%%%%%%%%%%%%%%%%%%%%%%%%%%%%%%%%%%%%%%%%%%%%





\question[10]
\textbf{True or False.}  
\begin{quote}
\emph{
Suppose $f:A\to B$ is a surjective map of sets.  Then there  exists an injective map of sets $s:B\to A$.  
}
\end{quote}
\textbf{  If true, give a proof.  If false, provide a counter example.}

\begin{proof} \ \\
\doublespacing
True.\\
Assume $ f: A \rightarrow B $ is a surjective map of sets. Then for all $ b \in B $ we have $ b = f(a), a \in A $. Assume $ c = f(a), d = f(b) $ where $ c, d \in B $ and $ a, b \in A $. Then there is a map of sets $ s : B \rightarrow A $ which $ s = f^{-1} $. Then we have $ s(c) = s(f(a)) = a $ and $ s(d) = s(f(b)) = b $. Assume $ a = b $, then we have $ f(a) = f(b) $, it follows that $ c = d $. There for $ s $ is injective.
\end{proof}


\prbbox{10}









%%%%%%%%%%%%%%%%%%%%%%%%%%%%%%%%%%%%%%%%%%%%%%%%%%%%%%%%%%%%%%%%%%%%%%%%%%%%%%%%%%




\question[10]
\textbf{True or False.}

\begin{quote}
  \emph{A map of sets $f:A\to B$ is surjective if and only if for all maps  of sets $g_1:B\to C$ and $g_2:B\to C$, we have that  $g_1\circ f=g_2\circ f$ implies $g_1=g_2$.} 
  \end{quote}
 \vskip .1 cm 
\textbf{  If true, give a proof.  If false, provide a counter example.}


\begin{proof}
\doublespacing
\ \\
True. \\
\textbf{Step1:} Prove if $ f: A \rightarrow B $ is surjective, then for all $ g_1: B \rightarrow C $ and $ g_2: B \rightarrow C $ we have that $ g_1 \circ f = g_2 \circ f \implies g_1 = g_2 $.\\
(contrapositive) Assume $ g_1 : B \rightarrow C, g_2 : B \rightarrow C $ and $ g_1 \neq g_2 $, then there exist $ g_1(b) \neq g_2(b), b \in B $ by definition. Suppose $ f: A \rightarrow B $ is surjective, then $ b = f(a) $ for some $ a \in A $. Thus we have $ g_1(f(a)) \neq g_2(f(a)) $, in other word $ g_1 \circ f \neq g_2 \circ f $.(contrapositive proof)\\ \\
\textbf{Step2:} Prove if for all maps of set $ g_1 : B \rightarrow C $ and $ g_2 : B \rightarrow C $, we have $ g_1 \circ f = g_2 \circ f $  implies $ g_1 = g_2 $, then $ f : A \rightarrow B $ is surjective. \\
(contrapositive) Assume a map of sets $ f : A \rightarrow B $. Suppose there exist $ g_1 : B \rightarrow C $ and $ g_2 : B \rightarrow C $ such that $ g_1 \circ f = g_2 \circ f $ and $ g_1 \neq g_2 $. We have $ g_1(f(a)) = g_2(f(a)) $ for all $ a \in A $. We also get $ g_1(b) \neq g_2(b) $ for some $ b \in B $. Thus it follows that $ b \neq f(a) $. Therefor $ f $ is not surjective by pigeonhole principle.

\end{proof}


\prbbox{10}


%%%%%%%%%%%%%%%%%%%%%%%%%%%%%%%%%%%%%%%%%%%%%%%%%%%%%%%%%%%%%%%%%%%%%%%%%%%%%%%%%%%%%%%%%%%



















\question   Define a relation on $\mathbb N\times \mathbb N$ by the rule that for all $(a,b)$, $(c,d) \in \mathbb N\times \mathbb N$, 
$$
(a,b)\sim (c,d)  \iff a+d=b+c.
$$ 


\begin{parts}


\part[5]  \textbf{What is this relation as a subset of $(\mathbb N\times \mathbb N)\times (\mathbb N\times \mathbb N)$?}

\begin{proof}
\doublespacing

$$ \{ ((a, b), (a, b)) | a, b, c, d \in N , a + d = b + c \} \subseteq (\mathbb{N} \times \mathbb{N}) \times (\mathbb{N} \times \mathbb{N}) $$

\end{proof}

\vspace{\stretch{1}}

\part[5] \textbf{Show that the relation is an equivalence relation.}  

\begin{proof}
\doublespacing

Assume $ (a, b), (c, d) \in \mathbb{N} \times \mathbb{N} $ and $ (a, b) \sim (c, d) $. Then we have $ a + d = b + c $ by definition, so $ c + b = d + a $. It follows that $ (c, d) \sim (a, b) $. Thus the relation is symetric. Suppose $ c = a $ and $ b = d $, then we have $ a + b = b + a $, it follows that $ (a, b) \sim (a, b) $. Thus it is reflexive. Assume $ (c, d) \sim (e, f) $, then we get $ c + f = d + e $, thus $ d - c = f - e $. We also get $ d - c = b - a $ from previous step, so $ f - e = b - a $, then $ a + f = b + e $, it follows that $ (a, b) \sim (e, f) $ by definition. Therefor the relation is transitive.

\vspace{\stretch{1}}

\end{proof}

\end{parts}







\prbbox{10}


%%%%%%%%%%%%%%%%%%%%%%%%%%%%%%%%%%%%%%%%%%%%%%%%%%%%%%%%%%%%%%%%%%%%%%%%%%%%%%%%%%%%%%%%%%%%%%%%%%%%%%%%%%%%%%%%%



\question[10]
\textbf{True or False.}   Let $Z=(\mathbb N\times \mathbb N)/\sim$ be the set of equivalence classes  in $\mathbb N\times \mathbb N$ for the equivalence relation given in the previous problem.  

\begin{quote}
\emph{There is a bijective map of sets
$$
f:Z\longrightarrow \mathbb Z
$$
given by the rule $[(a,b)]\mapsto a-b$.  
}
\end{quote}
\textbf{You must prove that the statement in italics is true or that it is false.}


\begin{proof}
\doublespacing

Ture \\
Assume the map of sets $ f : Z \rightarrow \mathbb{Z} $ where $ Z = ( \mathbb{N} \times \mathbb{N} ) / \sim $ is the set of equivalence classes, such that $ f( [ (a, b) ] ) = a - b, [ (a, b) ] \in Z $. Suppose the equivalence classes $ [ (a, b) ] $ and $ [ (c, d) ] $, such that $ f( [ (a, b) ] ) = f( [ (c, d) ] ) $, we have that $ a - b = c - d $, so $ a + d = b + c $, it follows that $ (a, b) \sim (c, d) $ and $ [ (a, b) ] = [ (c, d) ] $. There for $ f $ is injective. For every $ k \in \mathbb{Z} $, we can find a equivalence class $ [ (m + k, m) ] $, such that $ f( [ (m + k, m) ] ) = m + k - m = k $. Therefor $ f $ is surjective.

\end{proof}


\prbbox{10}







%%%%%%%%%%%%%%%%%%%%%%%%%%%%%%%%%%%%%%%%%%%%%%%%%%%%%%%%%%%%%%%%%%%%%%%%%%%%%%%%%%%%%%%%%%%%%%%%%%%%%%%%%%%%%%%%%


\question[10]



\textbf{True of False.}  Let $S$ be a set and   let $\mathscr C\subseteq \mathscr P(S)$ be a collection of subsets  of $S$ with the following property:   For any set $I$, if for each $i\in I$ we are given a set 
  $C_i\in \mathscr C$, and for all $i,j\in I$ we have $C_i\subseteq C_j$ or $C_j\subseteq C_i$, then  $\bigcup_{i\in I}C_i\in \mathscr C$.    
\begin{quote}
\emph{
There exists $C\in \mathscr C$ such that for all $C'\in \mathscr C$, if $C\subseteq C'$, then $C=C'$.
}
\end{quote}
\vskip .1 cm 
\textbf{If the statement in italics is true, give a proof.  Otherwise,  provide a counter example.}
\vskip .2 cm \noindent [Hint: Think in terms of POSETs.]



\begin{proof}
\doublespacing
True \\
Assume set $ I $ is the index set of $ \mathscr{C} $. Let $ C_i \in \mathscr{C} $, we have $ \{ C_i | i \in I \} $ are the elements of $ \mathscr{C} $. Suppose for all $ i, j \in I $ we have $ C_i \subseteq C_j $ or $ C_j \subseteq C_i $ and $ \bigcup_{i \in I} C_i \in \mathscr{C} $, then we get the partilly order relation $ R $, which $ aRb \Longleftrightarrow a \subseteq b $, and the poset $ (\mathscr{C}, \subseteq) $. Thus there is a element $ C \in \mathscr{C} $ such that $ C' \subseteq C $ for all $ C' \in \mathscr{C} $ by definition of poset. Suppose $ C \subseteq C'$, then we have $ C = C'$.


\end{proof}



\prbbox{10}
%%%%%%%%%%%%%%%%%%%%%%%%%%%%%%%%%%%%%%%%%%%%%%%%%%%%%%%%%%%%%%%%%%%%%%%%%%%%%%%



\question[10]
For an incidence plane $(\Pi,\Lambda)$, given a line $\ell\in \Lambda$ and a point $P\in \Pi$ such that $P\notin\ell$, consider the following properties:

\begin{quote}


\begin{enumerate}
\item[(P0)] There \emph{does not exist} $\ell'\in \Lambda$ such that $P\in \ell'$ and $\ell'\cap \ell = \emptyset$.

\item[(P1)] There \emph{exists exactly one} $\ell'\in \Lambda$ such that $P\in \ell'$ and $\ell'\cap \ell = \emptyset$.

\item[(P2)] There \emph{exists at least two}  $\ell'\in \Lambda$ such that $P\in \ell'$ and $\ell'\cap \ell = \emptyset$.

\end{enumerate}

\end{quote}


\textbf{Find an example of an incidence plane  $(\Pi,\Lambda)$ with exactly five points, such that for each of the properties above there exists a point and a line satisfying that property}.  In other words,  find a single   incidence plane $(\Pi,\Lambda)$ with $|\Pi|=5$, such that for $i=0,1,2$, there exists  $\ell_i\in \Lambda$  and $P_i\in \Pi$ with $P_i\notin \ell_i$, such that  $\ell_i$ and $P_i$ satisfy property (P$i$).







\prbbox{10}




%%%%%%%%%%%%%%%%%%%%%%%%%%%%%%%%%%%%%%%%%%%%%%%%%%%%%%%%%%%%%%%%%%%%%%%%%%%%%%%




\question
A polynomial in one variable, $x$,  with rational coefficients can, by definition, be written in the form
$$
p(x)= a_0+a_1x+a_2x^2+\cdots + a_dx^d
$$
where $a_0,a_1,\dots,a_d\in \mathbb Q$, and $d$ is some nonnegative integer.  If in the expression above for $p(x)$ we have $a_d\ne 0$, then we say $p(x)$ has degree $d$.  

\vskip .2 cm 
We denote by $\mathbb Q[x]$ the set of all polynomials in $x$ with rational coefficients.  Denote by $\mathbb Q[x]_d\subseteq \mathbb Q[x]$ the subset consisting of those polynomials that have degree at most $d$, union  the set containing the zero polynomial (the zero polynomial does not have a degree using our definition above). 

\begin{parts}
\part[5]  Let  $d$ be a nonnegative integer.  \textbf{Show that  the set $\mathbb Q[x]_d$ is countable}.  

\vskip .2 cm 
[Hint: Show that $\mathbb Q[x]_d$ is in bijection with $\mathbb Q^{d+1}$.]



\vspace{\stretch{2}}





\part[5] \textbf{True or False.} 
 \emph{The set $\mathbb Q[x]$ is countable.}  
\vskip .1 cm 
\textbf{You must prove that the statement in italics is true or that it is false.}





\vspace{\stretch{1}}


\end{parts}




\prbbox{10}



























\end{questions}
\end{document}			