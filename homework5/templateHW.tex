%%%%%%%%%%%%%%%%%%%%%%%%%%%%%%%%%%%%%%%%%%%% DOCUMENT CLASS %%%%%%%%%%%%%%%%%%%%%%%%%%%

\documentclass[12pt]{amsart}
%\documentclass[draft, 12pt]{amsart}


%%%%%%%%%%%%%%%%%%%%%%%%%%%%%%%%%%%%%%%%%%%%%%%%%% STANDARD PACKAGES %%%%%%%%%%%%%%%%%%%%%%%%%%%%%%%%%%%%%%%%%

\usepackage{amssymb,amsmath,amsthm,amscd,mathrsfs,graphicx, color}
\usepackage[cmtip,all,matrix,arrow,tips,curve]{xy}
\usepackage[notref,notcite]{showkeys}
%\usepackage[colorlinks]{hyperref}
\usepackage{multicol}
\usepackage{hyperref}
\usepackage[usenames,dvipsnames]{xcolor}
\hypersetup{colorlinks=true,citecolor=OliveGreen,linkcolor=BrickRed,urlcolor=BlueViolet}
\usepackage[active]{srcltx}
\usepackage{mathpazo}
\usepackage{setspace}\doublespacing 
%Double space, and make it easier for the grader to grade your homework.
%\usepackage{fullpage} 
%Use wide margins, and make it easier for the grader to grade your homework.


%%%%%%%%%%%%%%%%%%%%%%%%%%%%%%%%%%%%%%%%%%%%%% TIKZ FOR GRAPHING %%%%%%%%%%%%%%%%%%%%%%%%%%%%%%%%%%%%%%%%%%%%%%%%%%%%%%%
\usepackage{tikz,pgfplots}


%%%%%%%%%%%%%%%%%%%%%%%%%%%%%%%%%%%%%%%%%%%%%% 		ANSWER BOXES 		%%%%%%%%%%%%%%%%%%%%%%%%%%%%%%%%%%%%%%%%%%%
 \setlength\fboxsep{.3cm}
\setlength\fboxrule{.05cm}

\newcommand*{\boxedcolor}{red}
\makeatletter
\renewcommand{\boxed}[1]{\textcolor{\boxedcolor}{%
  \fbox{\normalcolor\m@th$\displaystyle#1$}}}
\makeatother

\makeatletter
\newcommand{\boxedred}[1]{\textcolor{red}{%
  \fbox{\normalcolor\m@th$\displaystyle#1$}}}
\makeatother

\makeatletter
\newcommand{\boxedblue}[1]{\textcolor{blue}{%
  \fbox{\normalcolor\m@th$\displaystyle#1$}}}
\makeatother




%%%%%%%%%%%%%%%%%%%%%%%%%%%%%%%%%%%%%%%%%%%%%%%%%%%% THEOREM ENVIRONMENTS %%%%%%%%%%%%%%%%%%%%%%%%%%%%%%%%


\numberwithin{equation}{section}
\newtheorem{teo}{Theorem}[section]
\newtheorem{pro}[teo]{Proposition}
\newtheorem{lem}[teo]{Lemma}
\newtheorem{cor}[teo]{Corollary}
\newtheorem{con}[teo]{Conjecture}
\newtheorem{convention}[teo]{}



\newtheorem{teoalpha}{Theorem}
\renewcommand{\theteoalpha}{\Alph{teoalpha}}
\newtheorem{proalpha}[teoalpha]{Proposition}
\newtheorem{coralpha}[teoalpha]{Corollary}


\theoremstyle{definition}
\newtheorem{dfn}[teo]{Definition}
\newtheorem{exa}[teo]{Example}
\newtheorem{que}[teo]{Question}

\theoremstyle{remark}
\newtheorem{rem}[teo]{Remark}
\newtheorem{nte}[teo]{Note}

%%%%%%%%%%%%%%%%%%%%%%%%%%%%%%%%%%%%%%%%%%%%%%% SOME CONDITIONALS FOR NOTES %%%%%%%%%%%%%%%%%%%%%%%%%%%%%%%%%%%

% Declare a new conditional
\newif\ifnotes
\notestrue 	% Show details
%\notesfalse 	% Exclude details


%%%%%%%%%%%%%%%%%%%%%%%%%%%%%%%%%%%%%%%%%%%%% COMMENTS %%%%%%%%%%%%%%%%%%%%%%%%%%%%

\newcommand{\marg}[1]{\normalsize{{\color{red}\footnote{{\color{blue}#1}}}{\marginpar[\vskip -.3cm {\color{BrickRed}\hfill\thefootnote$\implies$}]{\vskip -.3cm{ \color{BrickRed}$\impliedby$\thefootnote}}}}}

\newcommand{\qc}[1]{\marg{#1}}

%%%%%%%%%%%%%%%%%%%%%%%%%%%%%%%%%%%%%%%%%%%%%%%%%%% BEGIN DOCUMENT %%%%%%%%%%%%%%%%%%%%%%%%%%%%%%%%%%%
 
\begin{document}

\bibliographystyle{amsalpha}


%%%%%%%%%%%%%%%%%%%%%%%%%%%%%%%%%%%%%%%%%%%%%%%% AUTHOR INFO %%%%%%%%%%%%%%%%%%%%%%%%%%%%%%%%%%%% 

\author[QI]{QI WANG}
\address{University of Colorado, Department of Mathematics,  Campus Box 395,
Boulder, CO 80309-0395}
\email{casa@math.colorado.edu}
\date{\today}
%\thanks{I would like to take this opportunity to thank my class for their support.}


%%%%%%%%%%%%%%%%%%%%%%%%%%%%%%%%%%%%%%%%%%%%%%%% TITLE AND ABSTRACT %%%%%%%%%%%%%%%%%%%%%%%%%%%%%%%%%%%%

\title[Homework 5]{Homework 5 \\ \ \\  MATH 2001}

\begin{abstract} 
This is the first homework assignment.  The problems are from Hammack \cite[Ch.~2, \S 2.5]{H13}:
\begin{itemize}

\item \textbf{Chapter 2}  
\textbf{Section 2.5}, Exercises:  4, 6, 8.
\textbf{Section 2.6}, Exercises:  4, 6.
\textbf{Section 2.7}, Exercises:  2, 4, 8.

\end{itemize}
\end{abstract}


\maketitle


\tableofcontents

%%%%%%%%%%%%%%%%%%%%%%%%%%%%%%%%%%%%%%%%%%%%%%%% HOMEWORK ASSIGNMENT %%%%%%%%%%%%%%%%%%%%%%%%%%%%%%%%%%%%

%%%%%%%%%%%%%%%%%%%%%%%%%%%%%%%%%%%%%%%%%%%%%%%%%%%%%%%%%%%%%%%%%%%%%%%%%%%%%%%%%%%%%%%%%% CHAPTER 1 %%%%%%%%%%%%%%%%%%%%%%%%%%%%%%%%%%%%%%%%%%%%%%%%%%%%%%%%%%%%%%%%%%%%%%%%%%%%%%%%%%%%%%%%%%%%%%%%%%%


%%%%%%%%%%%%%%%%%%%%%%%%%%%%%%%%%%%%%%%%%%%%%%%% SECTION 1.1 %%%%%%%%%%%%%%%%%%%%%%%%%%%%%%%%%%%%

\section*{Chapter 1 Section 1.1}


%%%%%%%%%%%%%%%%%%%%%%%%%%%%%%%%%%%%%%%%%%%%%%%% EXERCISE 2 %%%%%%%%%%%%%%%%%%%%%%%%%%%%%%%%%%%%

\subsection*{Ch.2, \S 2.5,  Exercise 4, 6, 8}  Write a truth table for the logical statements.

\begin{enumerate}

\item[4.]
$ \lnot (\textbf{\textit{P}} \lor \textbf{\textit{Q}}) \lor (\lnot \textbf{\textit{P}}) $

\item[6.]
$ (\textbf{\textit{P}} \land \lnot \textbf{\textit{P}}) \land \textbf{\textit{Q}} $

\item[8.]
$ \textbf{\textit{P}} \lor (\textbf{\textit{Q}} \land \lnot \textbf{\textit{R}}) $

\end{enumerate}


\begin{proof}[Solution to Ch.2, \S 2.5,  Exercise 4, 6, 8] \ \\

\begin{enumerate}

\item[4.]
$ \lnot (\textbf{\textit{P}} \lor \textbf{\textit{Q}}) \lor (\lnot \textbf{\textit{P}}) $

\begin{displaymath}
\begin{array}{|c c c|c|c|c|}
P & Q & \lnot P & P \lor Q & \lnot (P \lor Q) & \lnot (P \lor Q) \lor (\lnot P) \\
T & T & F & T & F & F \\
T & F & F & T & F & F \\
F & T & T & T & F & T \\
F & F & T & F & T & T
\end{array}
\end{displaymath}

\item[6.]
$ (\textbf{\textit{P}} \land \lnot \textbf{\textit{P}}) \land \textbf{\textit{Q}} $

\begin{displaymath}
\begin{array}{|c c c|c|c|}
P & Q & \lnot P & P \land \lnot P & (P \land \lnot P) \land Q \\
T & T & F & F & F \\
T & F & F & F & F \\
F & T & T & F & F \\
F & F & T & F & F
\end{array}
\end{displaymath}

\item[8.]
$ \textbf{\textit{P}} \lor (\textbf{\textit{Q}} \land \lnot \textbf{\textit{R}}) $

\begin{displaymath}
\begin{array}{|c c c c|c|c|}
P & Q & R & \lnot R & Q \land \lnot R & P \lor (Q \land \lnot R)\\
T & T & T & F & F & T \\
T & T & F & T & T & T \\
T & F & T & F & F & T \\
T & F & F & T & F & T \\
F & T & T & F & F & F \\
F & T & F & T & T & T \\
F & F & T & F & F & F \\
F & F & F & T & F & F \\
\end{array}
\end{displaymath}

\end{enumerate}


\end{proof}


%%%%%%%%%%%%%%%%%%%%%%%%%%%%%%%%%%%%%%%%%%%%%%%% EXERCISE 8 %%%%%%%%%%%%%%%%%%%%%%%%%%%%%%%%%%%%


\subsection*{Ch.2, \S 2.6,  Exercise 4, 6}  Use truth tables to show that the following statements are logically equivalent. 

\begin{enumerate}

\item[4.]
$ \lnot (\textbf{\textit{P}} \lor \textbf{\textit{Q}}) = (\lnot \textbf{\textit{P}}) \land (\lnot \textbf{\textit{Q}}) $

\item[6.]
$ \lnot (\textbf{\textit{P}} \land \textbf{\textit{Q}} \land \textbf{\textit{R}}) = (\lnot \textbf{\textit{P}}) \lor (\lnot \textbf{\textit{Q}}) \lor (\lnot \textbf{\textit{R}}) $

\end{enumerate}


\begin{proof}[Solution to Ch.2, \S 2.6,  Exercise 4, 6] \ \\

\begin{enumerate}

\item[4.]
$ \lnot (\textbf{\textit{P}} \lor \textbf{\textit{Q}}) = (\lnot \textbf{\textit{P}}) \land (\lnot \textbf{\textit{Q}}) $

\begin{displaymath}
\begin{array}{|c c|c c|c c|c|}
P & Q & \lnot P & \lnot Q & (P \lor Q) & \lnot (P \lor Q) & (\lnot P) \land (\lnot Q) \\
T & T & F & F & T & F & F \\
T & F & F & T & T & F & F \\
F & T & T & F & T & F & F \\
F & F & T & T & F & T & T
\end{array}
\end{displaymath}

\item[6.]
$ \lnot (\textbf{\textit{P}} \land \textbf{\textit{Q}} \land \textbf{\textit{R}}) = (\lnot \textbf{\textit{P}}) \lor (\lnot \textbf{\textit{Q}}) \lor (\lnot \textbf{\textit{R}}) $

\begin{displaymath}
\begin{array}{|c c c|c c c|c c|c|}
P & Q & R & \lnot P & \lnot Q & \lnot R & P \land Q \land R & \lnot (P \land Q \land R) & (\lnot P) \lor (\lnot Q) \lor (\lnot R) \\
T & T & T & F & F & F & T & F & F \\
T & T & F & F & F & T & F & T & T \\
T & F & T & F & T & F & F & T & T \\
T & F & F & F & T & T & F & T & T \\
F & T & T & T & F & F & F & T & T \\
F & T & F & T & F & T & F & T & T \\
F & F & T & T & T & F & F & T & T \\
F & F & F & T & T & T & F & T & T
\end{array}
\end{displaymath}

\end{enumerate}

\end{proof}


%%%%%%%%%%%%%%%%%%%%%%%%%%%%%%%%%%%%%%%%%%%%%%%% EXERCISE 18 %%%%%%%%%%%%%%%%%%%%%%%%%%%%%%%%%%%%


\subsection*{Ch.2, \S 2.7,  Exercise 2, 4, 8} Write the following as English sentences. Say weather they are true or false.

\begin{enumerate}

\item[2.]
$ \forall x \in \mathbb{R}, \exists n \in \mathbb{N}, x^n \geq 0 $

\item[4.]
$ \forall \textbf{\textit{X}} \in \wp (\mathbb{N}), \textbf{\textit{X}} \subseteq \mathbb{R} $

\item[8.]
$ \forall n \in \mathbb{Z}, \exists \textbf{\textit{X}} \subseteq \mathbb{N}, |\textbf{\textit{X}}| = n $

\end{enumerate}


\begin{proof}[Solution to Ch.2, \S 2.7,  Exercise 18] \ \\

\begin{enumerate}

\item[2.]
$ \forall x \in \mathbb{R}, \exists n \in \mathbb{N}, x^n \geq 0 $
\ \\
For all x in the $ \mathbb{R} $, there exists n in $ \mathbb{N} $ such that $ x^n $ is greater than and equal to 0. \textbf{TRUE}

\item[4.]
$ \forall \textbf{\textit{X}} \in \wp(\mathbb{N}), \textbf{\textit{X}} \subseteq \mathbb{R} $
\ \\
All X in $ \wp (\mathbb{N}) $ is subsets of $ \mathbb{R} $. \textbf{FALSE}

\item[8.]
$ \forall n \in \mathbb{Z}, \exists \textbf{\textit{X}} \subseteq \mathbb{N}, |\textbf{\textit{X}}| = n $
\ \\
For all n in the $ \mathbb{Z} $, there exists $ \textbf{\textit{X}} $ in $ \mathbb{N} $ such that $ |\textbf{\textit{X}}| = n$. \textbf{FALSE}

\end{enumerate}

\end{proof}

\newpage

\ifnotes


\else
	This is not the full version.  This can be useful if there is scratch work you want to keep for yourself, but you do not want other people to see. 
\fi




\bibliography{templateHW}
\end{document}
