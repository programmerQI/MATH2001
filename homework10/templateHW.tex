%%%%%%%%%%%%%%%%%%%%%%%%%%%%%%%%%%%%%%%%%%%% DOCUMENT CLASS %%%%%%%%%%%%%%%%%%%%%%%%%%%

\documentclass[12pt]{amsart}
%\documentclass[draft, 12pt]{amsart}


%%%%%%%%%%%%%%%%%%%%%%%%%%%%%%%%%%%%%%%%%%%%%%%%%% STANDARD PACKAGES %%%%%%%%%%%%%%%%%%%%%%%%%%%%%%%%%%%%%%%%%

\usepackage{amssymb,amsmath,amsthm,amscd,mathrsfs,graphicx, color}
\usepackage[cmtip,all,matrix,arrow,tips,curve]{xy}
\usepackage[notref,notcite]{showkeys}
%\usepackage[colorlinks]{hyperref}
\usepackage{multicol}
\usepackage{hyperref}
\usepackage[usenames,dvipsnames]{xcolor}
\hypersetup{colorlinks=true,citecolor=OliveGreen,linkcolor=BrickRed,urlcolor=BlueViolet}
\usepackage[active]{srcltx}
\usepackage{mathpazo}
\usepackage{setspace}\doublespacing 
%Double space, and make it easier for the grader to grade your homework.
%\usepackage{fullpage} 
%Use wide margins, and make it easier for the grader to grade your homework.


%%%%%%%%%%%%%%%%%%%%%%%%%%%%%%%%%%%%%%%%%%%%%% TIKZ FOR GRAPHING %%%%%%%%%%%%%%%%%%%%%%%%%%%%%%%%%%%%%%%%%%%%%%%%%%%%%%%
\usepackage{tikz,pgfplots}


%%%%%%%%%%%%%%%%%%%%%%%%%%%%%%%%%%%%%%%%%%%%%% 		ANSWER BOXES 		%%%%%%%%%%%%%%%%%%%%%%%%%%%%%%%%%%%%%%%%%%%
 \setlength\fboxsep{.3cm}
\setlength\fboxrule{.05cm}

\newcommand*{\boxedcolor}{red}
\makeatletter
\renewcommand{\boxed}[1]{\textcolor{\boxedcolor}{%
  \fbox{\normalcolor\m@th$\displaystyle#1$}}}
\makeatother

\makeatletter
\newcommand{\boxedred}[1]{\textcolor{red}{%
  \fbox{\normalcolor\m@th$\displaystyle#1$}}}
\makeatother

\makeatletter
\newcommand{\boxedblue}[1]{\textcolor{blue}{%
  \fbox{\normalcolor\m@th$\displaystyle#1$}}}
\makeatother




%%%%%%%%%%%%%%%%%%%%%%%%%%%%%%%%%%%%%%%%%%%%%%%%%%%% THEOREM ENVIRONMENTS %%%%%%%%%%%%%%%%%%%%%%%%%%%%%%%%


\numberwithin{equation}{section}
\newtheorem{teo}{Theorem}[section]
\newtheorem{pro}[teo]{Proposition}
\newtheorem{lem}[teo]{Lemma}
\newtheorem{cor}[teo]{Corollary}
\newtheorem{con}[teo]{Conjecture}
\newtheorem{convention}[teo]{}



\newtheorem{teoalpha}{Theorem}
\renewcommand{\theteoalpha}{\Alph{teoalpha}}
\newtheorem{proalpha}[teoalpha]{Proposition}
\newtheorem{coralpha}[teoalpha]{Corollary}


\theoremstyle{definition}
\newtheorem{dfn}[teo]{Definition}
\newtheorem{exa}[teo]{Example}
\newtheorem{que}[teo]{Question}

\theoremstyle{remark}
\newtheorem{rem}[teo]{Remark}
\newtheorem{nte}[teo]{Note}

%%%%%%%%%%%%%%%%%%%%%%%%%%%%%%%%%%%%%%%%%%%%%%% SOME CONDITIONALS FOR NOTES %%%%%%%%%%%%%%%%%%%%%%%%%%%%%%%%%%%

% Declare a new conditional
\newif\ifnotes
\notestrue 	% Show details
%\notesfalse 	% Exclude details


%%%%%%%%%%%%%%%%%%%%%%%%%%%%%%%%%%%%%%%%%%%%% COMMENTS %%%%%%%%%%%%%%%%%%%%%%%%%%%%

\newcommand{\marg}[1]{\normalsize{{\color{red}\footnote{{\color{blue}#1}}}{\marginpar[\vskip -.3cm {\color{BrickRed}\hfill\thefootnote$\implies$}]{\vskip -.3cm{ \color{BrickRed}$\impliedby$\thefootnote}}}}}

\newcommand{\qc}[1]{\marg{#1}}

%%%%%%%%%%%%%%%%%%%%%%%%%%%%%%%%%%%%%%%%%%%%%%%%%%% BEGIN DOCUMENT %%%%%%%%%%%%%%%%%%%%%%%%%%%%%%%%%%%
 
\begin{document}

\bibliographystyle{amsalpha}


%%%%%%%%%%%%%%%%%%%%%%%%%%%%%%%%%%%%%%%%%%%%%%%% AUTHOR INFO %%%%%%%%%%%%%%%%%%%%%%%%%%%%%%%%%%%% 

\author[Casalaina]{Sebastian Casalaina}
\address{University of Colorado, Department of Mathematics,  Campus Box 395,
Boulder, CO 80309-0395}
\email{casa@math.colorado.edu}
\date{\today}
%\thanks{I would like to take this opportunity to thank my class for their support.}


%%%%%%%%%%%%%%%%%%%%%%%%%%%%%%%%%%%%%%%%%%%%%%%% TITLE AND ABSTRACT %%%%%%%%%%%%%%%%%%%%%%%%%%%%%%%%%%%%

\title[Homework 10]{Homework 10 \\ \ \\  MATH 2001}

\begin{abstract} 
This is the first homework assignment.  The problems are from Hammack \cite[Ch.~1, \S 1.1]{H13}:
\begin{itemize}

\item \textbf{Chapter 12}  
\textbf{Section 12.1}, Exercises:  4, 6.
\textbf{Section 12.2}, Exercises:  5, 10.
\textbf{Section 12.3}, Exercises:  1, 2.

\end{itemize}
\end{abstract}


\maketitle


\tableofcontents

%%%%%%%%%%%%%%%%%%%%%%%%%%%%%%%%%%%%%%%%%%%%%%%% HOMEWORK ASSIGNMENT %%%%%%%%%%%%%%%%%%%%%%%%%%%%%%%%%%%%

%%%%%%%%%%%%%%%%%%%%%%%%%%%%%%%%%%%%%%%%%%%%%%%%%%%%%%%%%%%%%%%%%%%%%%%%%%%%%%%%%%%%%%%%%% CHAPTER 1 %%%%%%%%%%%%%%%%%%%%%%%%%%%%%%%%%%%%%%%%%%%%%%%%%%%%%%%%%%%%%%%%%%%%%%%%%%%%%%%%%%%%%%%%%%%%%%%%%%%


%%%%%%%%%%%%%%%%%%%%%%%%%%%%%%%%%%%%%%%%%%%%%%%% SECTION 1.1 %%%%%%%%%%%%%%%%%%%%%%%%%%%%%%%%%%%%

\section*{Chapter 12 Section 12.1}


%%%%%%%%%%%%%%%%%%%%%%%%%%%%%%%%%%%%%%%%%%%%%%%% EXERCISE 2 %%%%%%%%%%%%%%%%%%%%%%%%%%%%%%%%%%%%

\subsection*{Ch.12, \S 12.1,  Exercise 4} There are eight different functions $ f : \{a, b, c \} \rightarrow \{0, 1 \} $. List them all. Diagrams will suffice.

\begin{proof}[Solution to Ch.1, \S 1.1,  Exercise 2] 
$$ f = \{(a, 0), (b, 0), (c, 0) \} $$
$$ f = \{(a, 0), (b, 1), (c, 0) \} $$
$$ f = \{(a, 0), (b, 0), (c, 1) \} $$
$$ f = \{(a, 0), (b, 1), (c, 1) \} $$
$$ f = \{(a, 1), (b, 0), (c, 0) \} $$
$$ f = \{(a, 1), (b, 1), (c, 0) \} $$
$$ f = \{(a, 1), (b, 0), (c, 1) \} $$
$$ f = \{(a, 1), (b, 1), (c, 1) \} $$
\end{proof}


%%%%%%%%%%%%%%%%%%%%%%%%%%%%%%%%%%%%%%%%%%%%%%%% EXERCISE 8 %%%%%%%%%%%%%%%%%%%%%%%%%%%%%%%%%%%%


\subsection*{Ch.12, \S 12.1,  Exercise 6}  Suppose $ f : \mathbb{Z} \rightarrow \mathbb{Z} $ is defined as $ f = \{(x, 4x + 5) : x \in \mathbb{Z} \} $. State the domain, codomain and range of $ f $. Find $ f(10) $.

\begin{proof}[Solution to Ch.12, \S 12.1,  Exercise 6] \ \\
The domain of the function is $ \{ x : x \in \mathbb{Z} \} $. \\
The codomain of the function is $ \{ x : x \in \mathbb{Z} \} $. \\
The range of the function is $ \{ x : x = 4n , n \in \mathbb{Z} \} $. \\

\end{proof}


%%%%%%%%%%%%%%%%%%%%%%%%%%%%%%%%%%%%%%%%%%%%%%%% EXERCISE 18 %%%%%%%%%%%%%%%%%%%%%%%%%%%%%%%%%%%%


\subsection*{Ch.12, \S 12.2,  Exercise 5}  A function $ f : \mathbb{Z} \rightarrow \mathbb{Z} $ is defined as $ f(n) = 2n + 1 $ . Verify whether this function is injective and whether it is surjective.

\begin{proof}[Solution to Ch.12, \S 12.2,  Exercise 5] \ \\
\textbf{Proposition.} The function is injective but not surjective.\\
\textbf{Step 1.} Prove the function is injective.
\textit{Proof.} Assume $ f(x) = f(y) $. We have $ 2x + 1 = 2y + 1 $, thus $ x = y $.\\
\textbf{Step 2.} Prove the function is not surjective.\\
\textit{Proof.} There exist element $ 2 $ which $ f(x) = 2x + 1 \neq 2 $ for every $ x \in \mathbb{Z} $. The function is injective but not surjective.
\end{proof}


%%%%%%%%%%%%%%%%%%%%%%%%%%%%%%%%%%%%%%%%%%%%%%%% EXERCISE 30 %%%%%%%%%%%%%%%%%%%%%%%%%%%%%%%%%%%%


\subsection*{Ch.12, \S 12.2,  Exercise 10}  Prove that the function $ f: \mathbb{R} - {1} \rightarrow \mathbb{R} - {1} $ defined by $ f(x) = \begin{pmatrix} \frac{x + 1}{x - 1} \end{pmatrix}^3 $ is bijective.

\begin{proof}[Solution to Ch.12, \S 12.2,  Exercise 10] \ \\
\textbf{Proposition.} The $ f(x) $ is bijective.\\
\textbf{Step 1.} Prove $ f(x) $ is injective.\\
\textit{Proof.} Assume $ f(x) = f(y) $, we have $ \begin{pmatrix} \frac{x + 1}{x - 1} \end{pmatrix}^3 = \begin{pmatrix} \frac{y + 1}{y - 1} \end{pmatrix}^3 $, taking the cubic root on both side get $ \begin{pmatrix} \frac{x + 1}{x - 1} \end{pmatrix} = \begin{pmatrix} \frac{y + 1}{y - 1} \end{pmatrix} $. Therefor we have:
$$ (x + 1) * (y - 1) = (y + 1) * (x - 1) $$
$$ xy + y - x = xy + x - y $$
$$ y - x = x - y $$
$$ 2y = 2x $$
$$ x = y $$
\textbf{Step 2.} Prove $ f(x) $ is sujective. \\
\textit{Proof.} Suppose $ f(x) = c $, we have $ f(x) = \begin{pmatrix} \frac{x + 1}{x - 1} \end{pmatrix}^3 = c $. Sove for $ x $, we get $ x = \frac{1 + \sqrt[3]{c}}{1 - \sqrt[3]{c}}, x \in \mathbb{R} $. It follows that $ f(x) $ is surjective.
The $ f(x) $ is surjective and injective, so it is bijective.
\end{proof}



%%%%%%%%%%%%%%%%%%%%%%%%%%%%%%%%%%%%%%%%%%%%%%%% EXERCISE 38 %%%%%%%%%%%%%%%%%%%%%%%%%%%%%%%%%%%%


\subsection*{Ch.12, \S 12.3,  Exercise 1}  Prove that if six numbers are chosen at random, then at least two of them will have the same remainder when divided by 5.

\begin{proof}[Solution to Ch.1, \S 1.1,  Exercise 38] Suppose the set $ A $ that made of the remanders of any number $ n $ where $ n \in \mathbb{Z} $ divided by 5. Such a set contains the elements $ A = \{0, 1, 2, 3 ,4 \} $. Thus, we have $ |A| = 5 $. The set $ B $ consist six random numbers, so $ |\textit{B}| = 6 $. Consider the function $ f : B \rightarrow A $, such that $ f(x) = x \ mod \  5 $. Because $ |B| > |A| $, the function is not injective by pigeohole principle.
\end{proof}


%%%%%%%%%%%%%%%%%%%%%%%%%%%%%%%%%%%%%%%%%%%%%%%% EXERCISE 40 %%%%%%%%%%%%%%%%%%%%%%%%%%%%%%%%%%%%


\subsection*{Ch.12, \S 12.3,  Exercise 2}  Prove that if $ a $ is a natural number, then there exist two unequal natural numbers $ k $ and $ l $ for which $ a^k - a^l $ is divisible by $ 10 $.

\begin{proof}[Solution to Ch.1, \S 1.1,  Exercise 40] Assume a set $ A = \{ x \in \mathbb{N} : x = a^n \} $ where $ a,n \in \mathbb{N} $. The $ a^k $ and $ a^l $ would be the two different elements of the set $ A $. Suppose the set $ B = \{ y \in \mathbb{N} : y = m \ mod \ 10 \} $, then $ |B| = 10 $. Consider a function $ f : A \rightarrow B $, such that $ f(z) = z \ mod \ 10 $. Based on pigeonhole princeple $ f $ is not injective, because $ |A| > |B| $. Therefor, there exist different $ k $ and $ l $, such that $ a^k \ mod \ 10 = a^l \ mod \ 10 $. Thus there exist $ (a^k - a^l) \div 10 = n $ where $ n \in \mathbb{N} $ based on definition.
\end{proof}





\ifnotes


\else
	This is not the full version.  This can be useful if there is scratch work you want to keep for yourself, but you do not want other people to see. 
\fi




\bibliography{templateHW}
\end{document}
