%%%%%%%%%%%%%%%%%%%%%%%%%%%%%%%%%%%%%%%%%%%% DOCUMENT CLASS %%%%%%%%%%%%%%%%%%%%%%%%%%%

\documentclass[12pt]{amsart}
%\documentclass[draft, 12pt]{amsart}


%%%%%%%%%%%%%%%%%%%%%%%%%%%%%%%%%%%%%%%%%%%%%%%%%% STANDARD PACKAGES %%%%%%%%%%%%%%%%%%%%%%%%%%%%%%%%%%%%%%%%%

\usepackage{amssymb,amsmath,amsthm,amscd,mathrsfs,graphicx, color}
\usepackage[cmtip,all,matrix,arrow,tips,curve]{xy}
\usepackage[notref,notcite]{showkeys}
%\usepackage[colorlinks]{hyperref}
\usepackage{multicol}
\usepackage{hyperref}
\usepackage[usenames,dvipsnames]{xcolor}
\hypersetup{colorlinks=true,citecolor=OliveGreen,linkcolor=BrickRed,urlcolor=BlueViolet}
\usepackage[active]{srcltx}
\usepackage{mathpazo}
\usepackage{setspace}\doublespacing 
%Double space, and make it easier for the grader to grade your homework.
%\usepackage{fullpage} 
%Use wide margins, and make it easier for the grader to grade your homework.


%%%%%%%%%%%%%%%%%%%%%%%%%%%%%%%%%%%%%%%%%%%%%% TIKZ FOR GRAPHING %%%%%%%%%%%%%%%%%%%%%%%%%%%%%%%%%%%%%%%%%%%%%%%%%%%%%%%
\usepackage{tikz,pgfplots}


%%%%%%%%%%%%%%%%%%%%%%%%%%%%%%%%%%%%%%%%%%%%%% 		ANSWER BOXES 		%%%%%%%%%%%%%%%%%%%%%%%%%%%%%%%%%%%%%%%%%%%
 \setlength\fboxsep{.3cm}
\setlength\fboxrule{.05cm}

\newcommand*{\boxedcolor}{red}
\makeatletter
\renewcommand{\boxed}[1]{\textcolor{\boxedcolor}{%
  \fbox{\normalcolor\m@th$\displaystyle#1$}}}
\makeatother

\makeatletter
\newcommand{\boxedred}[1]{\textcolor{red}{%
  \fbox{\normalcolor\m@th$\displaystyle#1$}}}
\makeatother

\makeatletter
\newcommand{\boxedblue}[1]{\textcolor{blue}{%
  \fbox{\normalcolor\m@th$\displaystyle#1$}}}
\makeatother




%%%%%%%%%%%%%%%%%%%%%%%%%%%%%%%%%%%%%%%%%%%%%%%%%%%% THEOREM ENVIRONMENTS %%%%%%%%%%%%%%%%%%%%%%%%%%%%%%%%


\numberwithin{equation}{section}
\newtheorem{teo}{Theorem}[section]
\newtheorem{pro}[teo]{Proposition}
\newtheorem{lem}[teo]{Lemma}
\newtheorem{cor}[teo]{Corollary}
\newtheorem{con}[teo]{Conjecture}
\newtheorem{convention}[teo]{}



\newtheorem{teoalpha}{Theorem}
\renewcommand{\theteoalpha}{\Alph{teoalpha}}
\newtheorem{proalpha}[teoalpha]{Proposition}
\newtheorem{coralpha}[teoalpha]{Corollary}


\theoremstyle{definition}
\newtheorem{dfn}[teo]{Definition}
\newtheorem{exa}[teo]{Example}
\newtheorem{que}[teo]{Question}

\theoremstyle{remark}
\newtheorem{rem}[teo]{Remark}
\newtheorem{nte}[teo]{Note}

%%%%%%%%%%%%%%%%%%%%%%%%%%%%%%%%%%%%%%%%%%%%%%% SOME CONDITIONALS FOR NOTES %%%%%%%%%%%%%%%%%%%%%%%%%%%%%%%%%%%

% Declare a new conditional
\newif\ifnotes
\notestrue 	% Show details
%\notesfalse 	% Exclude details


%%%%%%%%%%%%%%%%%%%%%%%%%%%%%%%%%%%%%%%%%%%%% COMMENTS %%%%%%%%%%%%%%%%%%%%%%%%%%%%

\newcommand{\marg}[1]{\normalsize{{\color{red}\footnote{{\color{blue}#1}}}{\marginpar[\vskip -.3cm {\color{BrickRed}\hfill\thefootnote$\implies$}]{\vskip -.3cm{ \color{BrickRed}$\impliedby$\thefootnote}}}}}

\newcommand{\qc}[1]{\marg{#1}}

%%%%%%%%%%%%%%%%%%%%%%%%%%%%%%%%%%%%%%%%%%%%%%%%%%% BEGIN DOCUMENT %%%%%%%%%%%%%%%%%%%%%%%%%%%%%%%%%%%
 
\begin{document}

\bibliographystyle{amsalpha}


%%%%%%%%%%%%%%%%%%%%%%%%%%%%%%%%%%%%%%%%%%%%%%%% AUTHOR INFO %%%%%%%%%%%%%%%%%%%%%%%%%%%%%%%%%%%% 

\author[Casalaina]{Sebastian Casalaina}
\address{University of Colorado, Department of Mathematics,  Campus Box 395,
Boulder, CO 80309-0395}
\email{casa@math.colorado.edu}
\date{\today}
%\thanks{I would like to take this opportunity to thank my class for their support.}


%%%%%%%%%%%%%%%%%%%%%%%%%%%%%%%%%%%%%%%%%%%%%%%% TITLE AND ABSTRACT %%%%%%%%%%%%%%%%%%%%%%%%%%%%%%%%%%%%

\title[Homework 3]{Homework 3 \\ \ \\  MATH 2001}

\begin{abstract} 
This is the third homework assignment.  The problems are from Hammack \cite[Ch.~1, \S 1.3, \S 1.4, \S 1.5, \S 1.6 \S 1.7, \S 1.8]{H13}:
\begin{itemize}

\item \textbf{Chapter 1}  
\textbf{Section 1.3}, Exercises:  2, 6, 10.

\item \textbf{Chapter 1}  
\textbf{Section 1.4}, Exercises:  2, 10, 14, 19.

\item \textbf{Chapter 1}  
\textbf{Section 1.5}, Exercises:  2, 4, 6.

\item \textbf{Chapter 1}  
\textbf{Section 1.6}, Exercises:  2.

\item \textbf{Chapter 1}  
\textbf{Section 1.7}, Exercises:  4, 5, 6.

\item \textbf{Chapter 1}  
\textbf{Section 1.8}, Exercises:  2, 11, 12.

\end{itemize}
\end{abstract}


\maketitle


\tableofcontents

%%%%%%%%%%%%%%%%%%%%%%%%%%%%%%%%%%%%%%%%%%%%%%%% HOMEWORK ASSIGNMENT %%%%%%%%%%%%%%%%%%%%%%%%%%%%%%%%%%%%

%%%%%%%%%%%%%%%%%%%%%%%%%%%%%%%%%%%%%%%%%%%%%%%%%%%%%%%%%%%%%%%%%%%%%%%%%%%%%%%%%%%%%%%%%% CHAPTER 1 %%%%%%%%%%%%%%%%%%%%%%%%%%%%%%%%%%%%%%%%%%%%%%%%%%%%%%%%%%%%%%%%%%%%%%%%%%%%%%%%%%%%%%%%%%%%%%%%%%%


%%%%%%%%%%%%%%%%%%%%%%%%%%%%%%%%%%%%%%%%%%%%%%%% SECTION 1.1 %%%%%%%%%%%%%%%%%%%%%%%%%%%%%%%%%%%%

\section*{Chapter 1 Section 1.3}


%%%%%%%%%%%%%%%%%%%%%%%%%%%%%%%%%%%%%%%%%%%%%%%% EXERCISE 2 %%%%%%%%%%%%%%%%%%%%%%%%%%%%%%%%%%%%

\subsection*{Ch.1, \S 1.3,  Exercise 2} List all the subsets of the following sets: $ \{1, 2, \emptyset \} $


\begin{proof}[Solution to Ch.1, \S 1.3,  Exercise 2] 

$$
\emptyset, \{ 1 \}, \{ 2 \}, \{ \emptyset \}, \{ 1, 2 \}, \{ 1, \emptyset \}, \{ 2, \emptyset \}, \{ 1, 2, \emptyset \}
$$

\end{proof}


%%%%%%%%%%%%%%%%%%%%%%%%%%%%%%%%%%%%%%%%%%%%%%%% EXERCISE 8 %%%%%%%%%%%%%%%%%%%%%%%%%%%%%%%%%%%%


\subsection*{Ch.1, \S 1.3,  Exercise 8}  List all the subsets of the following sets: $ \{1, 2, \emptyset \} $ 


\begin{proof}[Solution to Ch.1, \S 1.3,  Exercise 8]

$$
\emptyset, \{ \mathbb{R} \}, \{ \mathbb{Q} \}, \{ \mathbb{N} \}, \{ \mathbb{R}, \mathbb{Q} \}, \{ \mathbb{R}, \mathbb{N} \}, \{ \mathbb{Q}, \mathbb{N} \}, \{ \mathbb{R}, \mathbb{Q}, \mathbb{N} \}
$$

\end{proof}


%%%%%%%%%%%%%%%%%%%%%%%%%%%%%%%%%%%%%%%%%%%%%%%% EXERCISE 18 %%%%%%%%%%%%%%%%%%%%%%%%%%%%%%%%%%%%


\subsection*{Ch.1, \S 1.3,  Exercise 12}  Write out the following sets by listing their elements between braces. $ \{ \textbf{\textit{X}} : \textbf{\textit{X}} \subseteq \{ 3, 2, a, \} \ and \ |\textbf{\textit{X}}| = 1 \} $


\begin{proof}[Solution to Ch.1, \S 1.3,  Exercise 12]

$$
\{ 3 \}, \{ 2 \}, \{ a \}
$$

\end{proof}


%%%%%%%%%%%%%%%%%%%%%%%%%%%%%%%%%%%%%%%%%%%%%%%% EXERCISE 30 %%%%%%%%%%%%%%%%%%%%%%%%%%%%%%%%%%%%


\subsection*{Ch.1, \S 1.4,  Exercise 2}  Write out the following sets by listing their elements between braces. $ \mathscr{P}( \{ 1, 2, 3, 4 \} ) $



\begin{proof}[Solution to Ch.1, \S 1.4,  Exercise 2]

$$
\{ \emptyset, \{ 1 \}, \{ 2 \}, \{ 3 \}, \{ 4 \}, \{ 1, 2 \}, \{ 1, 3 \}, \{ 1, 4 \}, \{ 2, 3 \},
$$
$$
\{ 2, 4 \},\{ 3, 4 \}, \{ 1, 2, 3 \}, \{ 1, 2, 4 \}, \{ 2, 3, 4 \}, \{ 1, 2, 3, 4 \} \}
$$

\end{proof}



%%%%%%%%%%%%%%%%%%%%%%%%%%%%%%%%%%%%%%%%%%%%%%%% EXERCISE 38 %%%%%%%%%%%%%%%%%%%%%%%%%%%%%%%%%%%%


\subsection*{Ch.1, \S 1.4,  Exercise 10}  Write out the following sets by listing their elements between braces. $  \{ \textbf{\textit{X}} \in \mathscr{P}( \{ 1, 2, 3 \} ) : |\textbf{\textit{X}}| \leqslant 1 ) $



\begin{proof}[Solution to Ch.1, \S 1.4,  Exercise 10]

$$
\emptyset, \{ 1 \}, \{ 2 \}, \{ 3 \}
$$

\end{proof}


%%%%%%%%%%%%%%%%%%%%%%%%%%%%%%%%%%%%%%%%%%%%%%%% EXERCISE 40 %%%%%%%%%%%%%%%%%%%%%%%%%%%%%%%%%%%%


\subsection*{Ch.1, \S 1.4,  Exercise 14} Suppose that $ | \textbf{\textit{X}} | = m $ and $ | \textbf{\textit{X}} | = n $. Find the following cardinalities: $ | \mathscr{P}( \mathscr{P} ( \textbf{\textit{A}} ) ) | $ 


\begin{proof}[Solution to Ch.1, \S 1.4,  Exercise 14]

$$
| \mathscr{P}( \textbf{\textit{A}} ) | = 2^m
$$
$$
| \mathscr{P}( \mathscr{P} ( \textbf{\textit{A}} ) ) | = 2^{2^m}
$$

\end{proof}


%%%%%%%%%%%%%%%%%%%%%%%%%%%%%%%%%%%%%%%%%%%%%%%% EXERCISE 40 %%%%%%%%%%%%%%%%%%%%%%%%%%%%%%%%%%%%


\subsection*{Ch.1, \S 1.4,  Exercise 19} Suppose that $ | \textbf{\textit{X}} | = m $ and $ | \textbf{\textit{X}} | = n $. Find the following cardinalities: $ | \mathscr{P}( \mathscr{P} ( \mathscr{P}( A \times \emptyset ) ) ) | $ 


\begin{proof}[Solution to Ch.1, \S 1.4,  Exercise 19]

$$
| \textbf{\textit{A}} \times \emptyset | = \emptyset
$$
$$
| \mathscr{P}( \emptyset ) | = 2^0 = 1
$$
$$
| \mathscr{P}( \mathscr{P}( \emptyset ) ) | = 2^1 = 2
$$
$$
| \mathscr{P}( \mathscr{P} ( \mathscr{P} ( \emptyset ) ) | = 2^2 = 4
$$

\end{proof}


%%%%%%%%%%%%%%%%%%%%%%%%%%%%%%%%%%%%%%%%%%%%%%%% EXERCISE 40 %%%%%%%%%%%%%%%%%%%%%%%%%%%%%%%%%%%%


\subsection*{Ch.1, \S 1.5,  Exercise 2} Suppose that $ \textbf{\textit{A}} = \{0, 2, 4, 6, 8 \} $, $ \textbf{\textit{B}} = \{1, 3, 5, 7 \} $ and $ \textbf{\textit{C}} = \{2, 8, 4 \} $. Find:

\begin{enumerate}

\item[(a)]
$ \textbf{\textit{A}} \cup \textbf{\textit{B}} $
\item[(b)]
$ \textbf{\textit{A}} \cap \textbf{\textit{B}} $
\item[(c)]
$ \textbf{\textit{A}} - \textbf{\textit{B}} $
\item[(d)]
$ \textbf{\textit{A}} - \textbf{\textit{C}} $
\item[(e)]
$ \textbf{\textit{B}} - \textbf{\textit{A}} $
\item[(f)]
$ \textbf{\textit{A}} \cap \textbf{\textit{C}} $
\item[(g)]
$ \textbf{\textit{B}} \cap \textbf{\textit{C}} $
\item[(h)]
$ \textbf{\textit{C}} - \textbf{\textit{A}} $
\item[(i)]
$ \textbf{\textit{C}} - \textbf{\textit{B}} $

\end{enumerate} 


\begin{proof}[Solution to Ch.1, \S 1.5,  Exercise 2] \ \\

\begin{enumerate}
\item[(a)]
$ \textbf{\textit{A}} \cup \textbf{\textit{B}} $
$$
\textbf{\textit{A}} \cup \textbf{\textit{B}} = \{ 0, 1, 2, 3, 4, 5, 6, 7, 8 \}
$$

\item[(b)]
$ \textbf{\textit{A}} \cap \textbf{\textit{B}} $
$$
\textbf{\textit{A}} \cap \textbf{\textit{B}} = \emptyset
$$

\item[(c)]
$ \textbf{\textit{A}} - \textbf{\textit{B}} $
$$
\textbf{\textit{A}} - \textbf{\textit{B}} = \{ 0, 2, 4, 6, 8 \}
$$

\item[(d)]
$ \textbf{\textit{A}} - \textbf{\textit{C}} $
$$
\textbf{\textit{A}} - \textbf{\textit{C}} = \{ 0, 6 \}
$$

\item[(e)]
$ \textbf{\textit{B}} - \textbf{\textit{A}} $
$$
\textbf{\textit{B}} - \textbf{\textit{A}} = \emptyset
$$

\item[(f)]
$ \textbf{\textit{A}} \cap \textbf{\textit{C}} $
$$
\textbf{\textit{A}} \cap \textbf{\textit{C}} = \{ 2, 8, 4 \}
$$

\item[(g)]
$ \textbf{\textit{B}} \cap \textbf{\textit{C}} $
$$
\textbf{\textit{B}} \cap \textbf{\textit{C}} = \emptyset
$$

\item[(h)]
$ \textbf{\textit{C}} - \textbf{\textit{A}} $
$$
\textbf{\textit{C}} - \textbf{\textit{A}} = \emptyset
$$

\item[(i)]
$ \textbf{\textit{C}} - \textbf{\textit{B}} $
$$
\textbf{\textit{C}} - \textbf{\textit{B}} = \{ 2, 4, 8 \}
$$

\end{enumerate} 
\end{proof}


%%%%%%%%%%%%%%%%%%%%%%%%%%%%%%%%%%%%%%%%%%%%%%%% EXERCISE 40 %%%%%%%%%%%%%%%%%%%%%%%%%%%%%%%%%%%%


\subsection*{Ch.1, \S 1.5,  Exercise 4} Suppose that  $ \textbf{\textit{A}} = \{ a, b, c \} $, $ \textbf{\textit{B}} = \{ a, b \} $. Find:

\begin{enumerate}

\item[(a)]
$ ( \textbf{\textit{A}} \times \textbf{\textit{B}} ) \cap ( \textbf{\textit{B}} \times \textbf{\textit{B}} ) $

\item[(b)]
$ ( \textbf{\textit{A}} \times \textbf{\textit{B}} ) \cup ( \textbf{\textit{B}} \times \textbf{\textit{B}} ) $

\item[(c)]
$ ( \textbf{\textit{A}} \times \textbf{\textit{B}} ) - ( \textbf{\textit{B}} \times \textbf{\textit{B}} ) $

\item[(d)]
$ ( \textbf{\textit{A}} \cap \textbf{\textit{B}} ) \times \textbf{\textit{A}} $

\item[(e)]
$ ( \textbf{\textit{A}} \times \textbf{\textit{B}} ) \cap \textbf{\textit{B}} $

\item[(f)]
$ \mathscr{P}( \textbf{\textit{A}} ) \cap \mathscr{P}( \textbf{\textit{B}} ) $

\item[(g)]
$ \mathscr{P}( \textbf{\textit{A}} ) - \mathscr{P}( \textbf{\textit{B}} ) $

\item[(h)]
$ \mathscr{P}( \textbf{\textit{A}} \cap \textbf{\textit{B}} ) $

\item[(i)]
$ \mathscr{P}( \textbf{\textit{A}} ) \times \mathscr{P}( \textbf{\textit{B}} ) $

\end{enumerate}


\begin{proof}[Solution to Ch.1, \S 1.5,  Exercise 4] \ \\

\begin{enumerate}

\item[(a)]
$ ( \textbf{\textit{A}} \times \textbf{\textit{B}} ) \cap ( \textbf{\textit{B}} \times \textbf{\textit{B}} ) $
$$
\textbf{\textit{A}} \times \textbf{\textit{B}} = \{ (b, a), (b, b), (c, a), (c, b), (d, a), (d, b) \}
$$
$$
\textbf{\textit{B}} \times \textbf{\textit{B}} = \{ (a, a), (a, b), (b, a), (b, b) \}
$$
$$
( \textbf{\textit{A}} \times \textbf{\textit{B}} ) \cap ( \textbf{\textit{B}} \times \textbf{\textit{B}} ) = \{ (b, a), (b, b) \}
$$

\item[(b)]
$ ( \textbf{\textit{A}} \times \textbf{\textit{B}} ) \cup ( \textbf{\textit{B}} \times \textbf{\textit{B}} ) $
$$
( \textbf{\textit{A}} \times \textbf{\textit{B}} ) \cup ( \textbf{\textit{B}} \times \textbf{\textit{B}} ) = 
$$
$$
\{ (b, a), (b, b), (c, a), (c, b), (d, a), (d, b), (a, a), (a, b) \}
$$

\item[(c)]
$ ( \textbf{\textit{A}} \times \textbf{\textit{B}} ) - ( \textbf{\textit{B}} \times \textbf{\textit{B}} ) $
$$
( \textbf{\textit{A}} \times \textbf{\textit{B}} ) - ( \textbf{\textit{B}} \times \textbf{\textit{B}} ) = /{ (c, a), (c, b), (d, a), (d, b) /}
$$

\item[(d)]
$ ( \textbf{\textit{A}} \cap \textbf{\textit{B}} ) \times \textbf{\textit{A}} $
$$
\textbf{\textit{A}} \cap \textbf{\textit{B}} = \{ b \}
$$
$$
( \textbf{\textit{A}} \cap \textbf{\textit{B}} ) \times \textbf{\textit{A}} = \{ (b, a), (b, c), (b, d) \}
$$

\item[(e)]
$ ( \textbf{\textit{A}} \times \textbf{\textit{B}} ) \cap \textbf{\textit{B}} $
$$
\emptyset
$$

\item[(f)]
$ \mathscr{P}( \textbf{\textit{A}} ) \cap \mathscr{P}( \textbf{\textit{B}} ) $
$$
\mathscr{P}( \textbf{\textit{A}} ) = \{ \emptyset, \{ b \}, \{ c \}, \{ d \}, \{ b, c \}, \{ b, d \}, \{ c, d \}, \{ b, c, d \} \}
$$
$$
\mathscr{P}( \textbf{\textit{B}} ) = \{ \emptyset, \{ b \}, \{ a \}, \{ a, b \} \}
$$
$$
\mathscr{P}( \textbf{\textit{A}} ) \cap \mathscr{P}( \textbf{\textit{B}} ) = \{ \emptyset, \{ b \} \}
$$

\item[(g)]
$ \mathscr{P}( \textbf{\textit{A}} ) - \mathscr{P}( \textbf{\textit{B}} ) $
$$
 \mathscr{P}( \textbf{\textit{A}} ) - \mathscr{P}( \textbf{\textit{B}} ) = \{ \{ c \}, \{ d \}, \{ b, c \}, \{ b, d \}, \{ c, d \}, \{ b, c, d \} \}
$$

\item[(h)]
$ \mathscr{P}( \textbf{\textit{A}} \cap \textbf{\textit{B}} ) $
$$
\mathscr{P}( \textbf{\textit{A}} \cap \textbf{\textit{B}} ) = \{ \emptyset, \{ b \} \}
$$

\item[(i)]
$ \mathscr{P}( \textbf{\textit{A}} ) \times \mathscr{P}( \textbf{\textit{B}} ) $
$$
\mathscr{P}( \textbf{\textit{A}} ) \times \mathscr{P}( \textbf{\textit{B}} ) =
$$
$$
\emptyset, \emptyset \{ \{ b \}, \{ a \} \}, \{ \{ b \}, \{ b \} \}, \{ \{ b \}, \{ a, b \} \}, \emptyset, \{ \{ c \}, \{ a \} \}, \{ \{ c \}, \{ b \} \}, 
$$
$$
\{ \{ c \}, \{ a, b \} \}, \emptyset, \{ \{ d \}, \{ a \} \}, \{ \{ d \}, \{ b \} \}, \{ \{ d \}, \{ a, b \} \}, \emptyset, \{ \{ b, c \}, \{ a \} \}, 
$$
$$
\{ \{ b, c \}, \{ b \} \}, \{ \{ b, c \}, \{ a, b \} \}, \emptyset, \{ \{ b, d \}, \{ a \} \}, \{ \{ b, d \}, \{ b \} \},
$$
$$
\{ \{ b, d \}, \{ a, b \} \}, \emptyset, \{ \{ b, d \}, \{ a \} \}, \{ \{ b, d \}, \{ b \} \}, \{ \{ b, d \}, \{ a, b \} \}, \emptyset,
$$
$$
\{ \{ b, c, d \}, \{ a \} \}, \{ \{ b, c, d \}, \{ b \} \}, \{ \{ b, c, d \}, \{ a, b \} \} 
$$

\end{enumerate}

\end{proof}


%%%%%%%%%%%%%%%%%%%%%%%%%%%%%%%%%%%%%%%%%%%%%%%%%%%%%

\subsection*{Ch.1, \S 1.6,  Exercise 2} Let $ \textbf{\textit{A}} = \{ 0, 2, 4, 6, 8 \} $ and $ \textbf{\textit{B}} = \{ 1, 2, 5, 7 \} $ have universal set $ \textbf{\textit{U}} = \{ 0, 1, 2, \dots, 8 \} $. Find:

\begin{enumerate}
\item[(a)]
$ \overline{ \textbf{ \textit{A} } } $
\item[(b)]
$ \overline{ \textbf{ \textit{B} } } $
\item[(c)]
$ \textbf{ \textit{A} } \cap \overline{ \textbf{ \textit{A} } } $
\item[(d)]
$ \textbf{ \textit{A} } \cup \overline{ \textbf{ \textit{A} } } $
\item[(e)]
$ \textbf{ \textit{A} } - \overline{ \textbf{ \textit{A} } } $
\item[(f)]
$ \overline{ \textbf{ \textit{A} } \cup \textbf{ \textit{B} } } $
\item[(g)]
$ \overline{ \textbf{ \textit{A} } } \cap \overline{ \textbf{ \textit{B} } } $
\item[(h)]
$ \overline{ \textbf{ \textit{A} } \cap \textbf{ \textit{B} } } $
\item[(i)]
$ \overline{ \overline{ \textbf{ \textit{A} } } \cap \textbf{ \textit{B} } }
$

\end{enumerate}

\begin{proof}[Solution to Ch.1, \S 1.6,  Exercise 2] \ \\

\begin{enumerate}
\item[(a)]
$ \overline{ \textbf{ \textit{A} } } $
$$
\overline{ \textbf{ \textit{A} } } = \{ 1, 3, 5, 7 \}
$$
\item[(b)]
$ \overline{ \textbf{ \textit{B} } } $
$$
\overline{ \textbf{ \textit{B} } } = \{ 0, 3, 4, 6, 8 \}
$$
\item[(c)]
$ \textbf{ \textit{A} } \cap \overline{ \textbf{ \textit{A} } } $
$$
\textbf{ \textit{A} } \cap \overline{ \textbf{ \textit{A} } } = \emptyset
$$
\item[(d)]
$ \textbf{ \textit{A} } \cup \overline{ \textbf{ \textit{A} } } $
$$
\textbf{ \textit{A} } \cup \overline{ \textbf{ \textit{A} } } = \textbf{ \textit{U} }
$$
\item[(e)]
$ \textbf{ \textit{A} } - \overline{ \textbf{ \textit{A} } } $
$$
\textbf{ \textit{A} } - \overline{ \textbf{ \textit{A} } } = \textbf{ \textit{A} }
$$
\item[(f)]
$ \overline{ \textbf{ \textit{A} } \cup \textbf{ \textit{B} } } $
$$
\textbf{ \textit{A} } \cup \textbf{ \textit{B} } = \{ 0, 1, 2, 4, 5, 6, 7, 8 \}
$$
$$
\overline{ \textbf{ \textit{A} } \cup \textbf{ \textit{B} } } = \{ 3 \}
$$
\item[(g)]
$ \overline{ \textbf{ \textit{A} } } \cap \overline{ \textbf{ \textit{B} } } $
$$
\overline{ \textbf{ \textit{A} } } \cap \overline{ \textbf{ \textit{B} } } = \{ 3 \}
$$
\item[(h)]
$ \overline{ \textbf{ \textit{A} } \cap \textbf{ \textit{B} } } $
$$
\textbf{ \textit{A} } \cap \textbf{ \textit{B} } = \{ 2 \}
$$
$$
\overline{ \textbf{ \textit{A} } \cap \textbf{ \textit{B} } } = \{ 0, 1, 3, 4, 5, 6, 7, 8 \}
$$
\item[(i)]
$ \overline{ \overline{ \textbf{ \textit{A} } } \cap \textbf{ \textit{B} } }
$
$$
\overline{ \overline{ \textbf{ \textit{A} } } \cap \textbf{ \textit{B} } } = \{ 1, 5, 7 \}
$$

\end{enumerate}

\end{proof}


%%%%%%%%%%%%%%%%%%%%%%%%%%%%%%%%%%%%%%%%%%%%%%%%%%%%%

\subsection*{Ch.1, \S 1.7,  Exercise 4} Draw a Venn Diagram for $ ( \textbf{ \textit{A}} \cup \textbf{ \textit{B} } ) - \textbf{ \textit{C} }$.

\begin{proof}[Solution to Ch.1, \S 1.5,  Exercise 4] \ \\

\begin{center}

\def\firstcircle{(0,0) circle (1.5cm)}
\def\secondcircle{(45:2cm) circle (1.5cm)}
\def\thirdcircle{(0:2cm) circle (1.5cm)}
\begin{tikzpicture}
    \begin{scope} %[shift={(6cm,0cm)}]
        \begin{scope} %[even odd rule]% first circle without the second
            \clip \thirdcircle (-3,-3) rectangle (3,3);
        \fill[yellow] \firstcircle;
        \fill[yellow] \secondcircle;
        \end{scope}
        \draw \firstcircle node {$A$};
        \draw \secondcircle node {$B$};
        \draw \thirdcircle node {$C$};
    \end{scope}
\end{tikzpicture}

\end{center}

\end{proof}


%%%%%%%%%%%%%%%%%%%%%%%%%%%%%%%%%%%%%%%%%%%%%%%%%%%%%

\subsection*{Ch.1, \S 1.7,  Exercise 5} Draw a Venn Diagram for $ \textbf{ \textit{A}} \cup ( \textbf{ \textit{B} }  \cap \textbf{ \textit{C} } ) $ and $ ( \textbf{ \textit{A}} \cup \textbf{ \textit{B} } ) \cap ( \textbf{ \textit{A} } \cup \textbf{ \textit{C} } ) $. Base on your drawing, do you think $ \textbf{ \textit{A}} \cup ( \textbf{ \textit{B} }  \cap \textbf{ \textit{C} } ) = ( \textbf{ \textit{A}} \cup \textbf{ \textit{B} } ) \cap ( \textbf{ \textit{A} } \cup \textbf{ \textit{C} } ) $?

\begin{proof}[Solution to Ch.1, \S 1.7,  Exercise 5] \ \\
$$
 \textbf{ \textit{A}} \cup ( \textbf{ \textit{B} }  \cap \textbf{ \textit{C} } )
$$

\begin{center}
\def\firstcircle{(0,0) circle (1.5cm)}
\def\secondcircle{(45:2cm) circle (1.5cm)}
\def\thirdcircle{(0:2cm) circle (1.5cm)}
\begin{tikzpicture}
    \begin{scope}
        \begin{scope}
        \clip \thirdcircle;
        \fill[yellow] \secondcircle;
        \end{scope}
        \begin{scope}
        \fill[yellow] \firstcircle;
        \end{scope}
        \draw \firstcircle node {$A$};
        \draw \secondcircle node {$B$};
        \draw \thirdcircle node {$C$};
    \end{scope}
\end{tikzpicture}
\end{center}

$$
( \textbf{ \textit{A}} \cup \textbf{ \textit{B} } ) \cap ( \textbf{ \textit{A} } \cup \textbf{ \textit{C} } ) 
$$

\begin{center}

\def\firstcircle{(0,0) circle (1.5cm)}
\def\secondcircle{(45:2cm) circle (1.5cm)}
\def\thirdcircle{(0:2cm) circle (1.5cm)}
\begin{tikzpicture}
    \begin{scope}
        \begin{scope}
        \clip \thirdcircle;
        \fill[yellow] \secondcircle;
        \end{scope}
        \begin{scope}
        \fill[yellow] \firstcircle;
        \end{scope}
        \draw \firstcircle node {$A$};
        \draw \secondcircle node {$B$};
        \draw \thirdcircle node {$C$};
    \end{scope}
\end{tikzpicture}

\end{center}

\end{proof}


%%%%%%%%%%%%%%%%%%%%%%%%%%%%%%%%%%%%%%%%%%%%%%%%%%%%%

\subsection*{Ch.1, \S 1.7,  Exercise 6}  Draw a Venn Diagram for $ \textbf{ \textit{A}} \cap ( \textbf{ \textit{B} }  \cup \textbf{ \textit{C} } ) $ and $ ( \textbf{ \textit{A}} \cap \textbf{ \textit{B} } ) \cup ( \textbf{ \textit{A} } \cap \textbf{ \textit{C} } ) $. Base on your drawing, do you think $ \textbf{ \textit{A}} \cap ( \textbf{ \textit{B} }  \cup \textbf{ \textit{C} } ) = ( \textbf{ \textit{A}} \cap \textbf{ \textit{B} } ) \cup ( \textbf{ \textit{A} } \cap \textbf{ \textit{C} } ) $?

\begin{proof}[Solution to Ch.1, \S 1.7,  Exercise 6] \ \\
$$
\textbf{ \textit{A}} \cap ( \textbf{ \textit{B} }  \cup \textbf{ \textit{C} } )
$$

\begin{center}

\def\firstcircle{(0,0) circle (1.5cm)}
\def\secondcircle{(45:2cm) circle (1.5cm)}
\def\thirdcircle{(0:2cm) circle (1.5cm)}
\begin{tikzpicture}
    \begin{scope}
        \begin{scope}
        \clip \firstcircle;
        \fill[yellow] \secondcircle;
        \fill[yellow] \thirdcircle;
        \end{scope}
        \draw \firstcircle node {$A$};
        \draw \secondcircle node {$B$};
        \draw \thirdcircle node {$C$};
    \end{scope}
\end{tikzpicture}

\end{center}

$$
( \textbf{ \textit{A}} \cap \textbf{ \textit{B} } ) \cup ( \textbf{ \textit{A} } \cap \textbf{ \textit{C} } )
$$

\begin{center}

\def\firstcircle{(0,0) circle (1.5cm)}
\def\secondcircle{(45:2cm) circle (1.5cm)}
\def\thirdcircle{(0:2cm) circle (1.5cm)}
\begin{tikzpicture}
    \begin{scope}
        \begin{scope}
        \clip \firstcircle;
        \fill[yellow] \secondcircle;
        \fill[yellow] \thirdcircle;
        \end{scope}
        \draw \firstcircle node {$A$};
        \draw \secondcircle node {$B$};
        \draw \thirdcircle node {$C$};
    \end{scope}
\end{tikzpicture}

\end{center}

\end{proof}


%%%%%%%%%%%%%%%%%%%%%%%%%%%%%%%%%%%%%%%%%%%%%%%%%%%

\subsection*{Ch.1, \S 1.8,  Exercise 2} Suppose
$
\begin{cases}
      & \textbf{ \textit{A} }_1 \ = \ \{ 0, 2, 4, 8, 10, 12, 14, 16, 18, 20, 22, 24 \}, \\
      & \textbf{ \textit{A} }_2 \ = \ \{ 0, 3, 6, 9, 12, 15, 21, 24 \}, \\
      & \textbf{ \textit{A} }_3 \ = \ \{ 0, 4, 8, 12, 16, 20, 24 \}.
\end{cases}
$

\begin{enumerate}
\item[(a)]
$ \bigcup\limits_{ i = 1 }^{ 3 } \textbf{ \textit{A} }_{ i } $
\item[(b)]
$ \bigcap\limits_{ i = 1 }^{ 3 } \textbf{ \textit{A} }_{ i } $

\end{enumerate}

\begin{proof}[Solution to Ch.1, \S 1.8,  Exercise 2] \ \\

\begin{enumerate}
\item[(a)]
$ \bigcup\limits_{ i = 1 }^{ 3 } \textbf{ \textit{A} }_{ i } $
$$
\{ 0, 2, 3, 4, 6, 8, 9, 10, 12, 14, 15, 16, 18, 20, 21, 22, 24 \}
$$

\item[(b)]
$ \bigcap\limits_{ i = 1 }^{ 3 } \textbf{ \textit{A} }_{ i } $
$$
\{ 0, 12, 24 \}
$$

\end{enumerate}

\end{proof}


%%%%%%%%%%%%%%%%%%%%%%%%%%%%%%%%%%%%%%%%%%%%%%%%%%%%

\subsection*{Ch.1, \S 1.8,  Exercise 11} $ \bigcup\limits_{ \alpha \in \textit{I} } \textbf{ \textit{A} }_{ \alpha } \subseteq \bigcap\limits_{ \alpha \in \textit{I} } \textbf{ \textit{A} }_{ \alpha } $ always true for any collection of sets $ \textbf{ \textit{A} }_{ \alpha } $ with index set $ \textbf{ \textit{I} } $?

\begin{proof}[Solution to Ch.1, \S 1.8,  Exercise 11] \ \\

It is always true that $ \bigcup\limits_{ \alpha \in \textit{I} } \textbf{ \textit{A} }_{ \alpha } \subseteq \textbf{ \textit{A} }_{ \alpha } $ and $ \textbf{ \textit{A} }_{ \alpha } \subseteq \bigcap\limits_{ \alpha \in \textit{I} } \textbf{ \textit{A} }_{ \alpha } $, so $ \bigcup\limits_{ \alpha \in \textit{I} } \textbf{ \textit{A} }_{ \alpha } \subseteq \bigcap\limits_{ \alpha \in \textit{I} } \textbf{ \textit{A} }_{ \alpha } $ is always true.

\end{proof}


%%%%%%%%%%%%%%%%%%%%%%%%%%%%%%%%%%%%%%%%%%%%%%%%%%%%

\subsection*{Ch.1, \S 1.8,  Exercise 12} $ \bigcup\limits_{ \alpha \in \textit{I} } \textbf{ \textit{A} }_{ \alpha } = \bigcap\limits_{ \alpha \in \textit{I} } \textbf{ \textit{A} }_{ \alpha } $, what do you think can be said about the relationships between the sets $\textbf{ \textit{A} }_{ \alpha }$?

\begin{proof}[Solution to Ch.1, \S 1.5,  Exercise 4] \ \\

They are all equal.

\end{proof}




\ifnotes


\else
	This is not the full version.  This can be useful if there is scratch work you want to keep for yourself, but you do not want other people to see. 
\fi




\bibliography{templateHW}
\end{document}
