%%%%%%%%%%%%%%%%%%%%%%%%%%%%%%%%%%%%%%%%%%%% DOCUMENT CLASS %%%%%%%%%%%%%%%%%%%%%%%%%%%

\documentclass[12pt]{amsart}
%\documentclass[draft, 12pt]{amsart}


%%%%%%%%%%%%%%%%%%%%%%%%%%%%%%%%%%%%%%%%%%%%%%%%%% STANDARD PACKAGES %%%%%%%%%%%%%%%%%%%%%%%%%%%%%%%%%%%%%%%%%

\usepackage{amssymb,amsmath,amsthm,amscd,mathrsfs,graphicx, color}
\usepackage[cmtip,all,matrix,arrow,tips,curve]{xy}
\usepackage[notref,notcite]{showkeys}
%\usepackage[colorlinks]{hyperref}
\usepackage{multicol}
\usepackage{hyperref}
\usepackage[usenames,dvipsnames]{xcolor}
\hypersetup{colorlinks=true,citecolor=OliveGreen,linkcolor=BrickRed,urlcolor=BlueViolet}
\usepackage[active]{srcltx}
\usepackage{mathpazo}
\usepackage{setspace}\doublespacing 
%Double space, and make it easier for the grader to grade your homework.
%\usepackage{fullpage} 
%Use wide margins, and make it easier for the grader to grade your homework.


%%%%%%%%%%%%%%%%%%%%%%%%%%%%%%%%%%%%%%%%%%%%%% TIKZ FOR GRAPHING %%%%%%%%%%%%%%%%%%%%%%%%%%%%%%%%%%%%%%%%%%%%%%%%%%%%%%%
\usepackage{tikz,pgfplots}


%%%%%%%%%%%%%%%%%%%%%%%%%%%%%%%%%%%%%%%%%%%%%% 		ANSWER BOXES 		%%%%%%%%%%%%%%%%%%%%%%%%%%%%%%%%%%%%%%%%%%%
 \setlength\fboxsep{.3cm}
\setlength\fboxrule{.05cm}

\newcommand*{\boxedcolor}{red}
\makeatletter
\renewcommand{\boxed}[1]{\textcolor{\boxedcolor}{%
  \fbox{\normalcolor\m@th$\displaystyle#1$}}}
\makeatother

\makeatletter
\newcommand{\boxedred}[1]{\textcolor{red}{%
  \fbox{\normalcolor\m@th$\displaystyle#1$}}}
\makeatother

\makeatletter
\newcommand{\boxedblue}[1]{\textcolor{blue}{%
  \fbox{\normalcolor\m@th$\displaystyle#1$}}}
\makeatother




%%%%%%%%%%%%%%%%%%%%%%%%%%%%%%%%%%%%%%%%%%%%%%%%%%%% THEOREM ENVIRONMENTS %%%%%%%%%%%%%%%%%%%%%%%%%%%%%%%%


\numberwithin{equation}{section}
\newtheorem{teo}{Theorem}[section]
\newtheorem{pro}[teo]{Proposition}
\newtheorem{lem}[teo]{Lemma}
\newtheorem{cor}[teo]{Corollary}
\newtheorem{con}[teo]{Conjecture}
\newtheorem{convention}[teo]{}



\newtheorem{teoalpha}{Theorem}
\renewcommand{\theteoalpha}{\Alph{teoalpha}}
\newtheorem{proalpha}[teoalpha]{Proposition}
\newtheorem{coralpha}[teoalpha]{Corollary}


\theoremstyle{definition}
\newtheorem{dfn}[teo]{Definition}
\newtheorem{exa}[teo]{Example}
\newtheorem{que}[teo]{Question}

\theoremstyle{remark}
\newtheorem{rem}[teo]{Remark}
\newtheorem{nte}[teo]{Note}

%%%%%%%%%%%%%%%%%%%%%%%%%%%%%%%%%%%%%%%%%%%%%%% SOME CONDITIONALS FOR NOTES %%%%%%%%%%%%%%%%%%%%%%%%%%%%%%%%%%%

% Declare a new conditional
\newif\ifnotes
\notestrue 	% Show details
%\notesfalse 	% Exclude details


%%%%%%%%%%%%%%%%%%%%%%%%%%%%%%%%%%%%%%%%%%%%% COMMENTS %%%%%%%%%%%%%%%%%%%%%%%%%%%%

\newcommand{\marg}[1]{\normalsize{{\color{red}\footnote{{\color{blue}#1}}}{\marginpar[\vskip -.3cm {\color{BrickRed}\hfill\thefootnote$\implies$}]{\vskip -.3cm{ \color{BrickRed}$\impliedby$\thefootnote}}}}}

\newcommand{\qc}[1]{\marg{#1}}

%%%%%%%%%%%%%%%%%%%%%%%%%%%%%%%%%%%%%%%%%%%%%%%%%%% BEGIN DOCUMENT %%%%%%%%%%%%%%%%%%%%%%%%%%%%%%%%%%%
 
\begin{document}

\bibliographystyle{amsalpha}


%%%%%%%%%%%%%%%%%%%%%%%%%%%%%%%%%%%%%%%%%%%%%%%% AUTHOR INFO %%%%%%%%%%%%%%%%%%%%%%%%%%%%%%%%%%%% 

\author[QI WANG]{QI WANG}
\address{University of Colorado, Department of Mathematics,  Campus Box 395,
Boulder, CO 80309-0395}
\email{casa@math.colorado.edu}
\date{\today}
%\thanks{I would like to take this opportunity to thank my class for their support.}


%%%%%%%%%%%%%%%%%%%%%%%%%%%%%%%%%%%%%%%%%%%%%%%% TITLE AND ABSTRACT %%%%%%%%%%%%%%%%%%%%%%%%%%%%%%%%%%%%

\title[Homework 4]{Homework 4 \\ \ \\  MATH 2001}

\begin{abstract} 
This is the first homework assignment.  The problems are from Hammack \cite[Ch.~2]{H13}:
\begin{itemize}

\item \textbf{Chapter 2}  
\textbf{Section 2.1}, Exercises:  2, 4, 6.
\textbf{Section 2.2}, Exercises:  2, 6.
\textbf{Section 2.3}, Exercises:  8, 10.
\textbf{Section 2.4}, Exercises:  4.



\end{itemize}
\end{abstract}


\maketitle


\tableofcontents

%%%%%%%%%%%%%%%%%%%%%%%%%%%%%%%%%%%%%%%%%%%%%%%% HOMEWORK ASSIGNMENT %%%%%%%%%%%%%%%%%%%%%%%%%%%%%%%%%%%%

%%%%%%%%%%%%%%%%%%%%%%%%%%%%%%%%%%%%%%%%%%%%%%%%%%%%%%%%%%%%%%%%%%%%%%%%%%%%%%%%%%%%%%%%%% CHAPTER 1 %%%%%%%%%%%%%%%%%%%%%%%%%%%%%%%%%%%%%%%%%%%%%%%%%%%%%%%%%%%%%%%%%%%%%%%%%%%%%%%%%%%%%%%%%%%%%%%%%%%


%%%%%%%%%%%%%%%%%%%%%%%%%%%%%%%%%%%%%%%%%%%%%%%% SECTION 1.1 %%%%%%%%%%%%%%%%%%%%%%%%%%%%%%%%%%%%

\section*{Chapter 2 Section 2.1}


%%%%%%%%%%%%%%%%%%%%%%%%%%%%%%%%%%%%%%%%%%%%%%%% EXERCISE 2 %%%%%%%%%%%%%%%%%%%%%%%%%%%%%%%%%%%%

\subsection*{Ch.2, \S 2.1,  Exercise 2}  Decide whether or not the following are statements. In case of a statment, say if it is true or false, if possible: "Every even integer is a real number."


\begin{proof}[Solution to Ch.1, \S 2.1,  Exercise 2] 
\ \\

\begin{center}
It is a statement. \\
It is true.
\end{center}

\end{proof}


%%%%%%%%%%%%%%%%%%%%%%%%%%%%%%%%%%%%%%%%%%%%%%%% EXERCISE 8 %%%%%%%%%%%%%%%%%%%%%%%%%%%%%%%%%%%%


\subsection*{Ch.2, \S 2.1,  Exercise 4}  Decide whether or not the following are statements. In case of a statment, say if it is true or false, if possible: "Set $ \mathbb{Z} $ and set $ \mathbb{N} $"


\begin{proof}[Solution to Ch.1, \S 1.1,  Exercise 8]
\ \\

\begin{center}
It is not a statement.
\end{center}

\end{proof}


%%%%%%%%%%%%%%%%%%%%%%%%%%%%%%%%%%%%%%%%%%%%%%%% EXERCISE 18 %%%%%%%%%%%%%%%%%%%%%%%%%%%%%%%%%%%%


\subsection*{Ch.2, \S 2.1,  Exercise 6}  



\begin{proof}[Solution to Ch.1, \S 2.1,  Exercise 6]
\ \\

\begin{center}
It is a statement. \\
It is true.
\end{center}

\end{proof}


%%%%%%%%%%%%%%%%%%%%%%%%%%%%%%%%%%%%%%%%%%%%%%%% EXERCISE 30 %%%%%%%%%%%%%%%%%%%%%%%%%%%%%%%%%%%%


\subsection*{Ch.2, \S 2.2,  Exercise 2}  Express each statement or open sentence in one of the forms $ \textit{P} \lor \textit{Q} $, $ \textit{P} \land \textit{Q} $, or $ \lnot \textit{P} $. Be sure to also state what statements $ \textit{P} $ and $ \textit{Q} $ stand for: "The matrix A is not invertible."



\begin{proof}[Solution to Ch.2, \S 2.2,  Exercise 2]
\ \\

\begin{center}
$ \textit{P} : $ The matrix A is invertible. \\
$ \lnot \textit{P} $
\end{center}

\end{proof}



%%%%%%%%%%%%%%%%%%%%%%%%%%%%%%%%%%%%%%%%%%%%%%%% EXERCISE 38 %%%%%%%%%%%%%%%%%%%%%%%%%%%%%%%%%%%%


\subsection*{Ch.2, \S 2.2,  Exercise 6}   Express each statement or open sentence in one of the forms $ \textit{P} \lor \textit{Q} $, $ \textit{P} \land \textit{Q} $, or $ \lnot \textit{P} $. Be sure to also state what statements $ \textit{P} $ and $ \textit{Q} $ stand for: "There is a quiz scheduled for Wednesday or Friday."



\begin{proof}[Solution to Ch.2, \S 2.2,  Exercise 6]
\ \\

\begin{center}
$ \textit{P} $: There is a quiz scheduled for Wednesday. \\
$ \textit{Q} $: There is a quiz scheduled for Friday. \\
$ \textit{P} \land \textit{Q} $
\end{center}

\end{proof}


%%%%%%%%%%%%%%%%%%%%%%%%%%%%%%%%%%%%%%%%%%%%%%%% EXERCISE 40 %%%%%%%%%%%%%%%%%%%%%%%%%%%%%%%%%%%%


\subsection*{Ch.2, \S 2.3,  Exercise 8}  Without changing their meanings, convert each of the following sentences into sentence having the form "$ \textit{ If P, then Q } $": "A geometric series with ratio $ \textit{r} $ converges if $ | \textit{r} | < 1 $."S


\begin{proof}[Solution to Ch.2, \S 2.2,  Exercise 8]
\ 
\\ 

\begin{center}
If $ | \textit{r} | < 1 $, then the geometric series with ratio $ \textit{r} $ converges.
\end{center}

\end{proof}


%%%%%%%%%%%%%%%%%%%%%%%%%%%%%%%%%%%%%%%%%%%%%%%% EXERCISE 40 %%%%%%%%%%%%%%%%%%%%%%%%%%%%%%%%%%%%


\subsection*{Ch.2, \S 2.3,  Exercise 8}  Without changing their meanings, convert each of the following sentences into sentence having the form "$ \textit{ If P, then Q } $": "The discriminant is negative only if the quadratic equation has no real solutions."


\begin{proof}[Solution to Ch.2, \S 2.2,  Exercise 8]
\ 
\\ 

\begin{center}
If the discriminant is negative, then the quadratic equation has no real solutions.
\end{center}

\end{proof}


%%%%%%%%%%%%%%%%%%%%%%%%%%%%%%%%%%%%%%%%%%%%%%%% EXERCISE 40 %%%%%%%%%%%%%%%%%%%%%%%%%%%%%%%%%%%%


\subsection*{Ch.2, \S 2.3,  Exercise 8}  Without changing their meanings, convert each of the following sentences into sentence having the form "$ \textit{ P if and only if Q } $": "If $ \textit{a} \in \mathbb{Q} $ then $ 5 \textit{a} \in \mathbb{Q} $, and if $ 5 \textit{a} \in \mathbb{Q} $ then $ \textit{a} \in \mathbb{Q} $. "


\begin{proof}[Solution to Ch.2, \S 2.2,  Exercise 8]
\ 
\\ 

\begin{center}
$ \textit{a} \in \mathbb{Q} $ if and only if $ 5 \textit{a} \in \mathbb{Q} $.
\end{center}

\end{proof}





\ifnotes


\else
	This is not the full version.  This can be useful if there is scratch work you want to keep for yourself, but you do not want other people to see. 
\fi




\bibliography{templateHW}
\end{document}
