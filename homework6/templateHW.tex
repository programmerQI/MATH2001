%%%%%%%%%%%%%%%%%%%%%%%%%%%%%%%%%%%%%%%%%%%% DOCUMENT CLASS %%%%%%%%%%%%%%%%%%%%%%%%%%%

\documentclass[12pt]{amsart}
%\documentclass[draft, 12pt]{amsart}


%%%%%%%%%%%%%%%%%%%%%%%%%%%%%%%%%%%%%%%%%%%%%%%%%% STANDARD PACKAGES %%%%%%%%%%%%%%%%%%%%%%%%%%%%%%%%%%%%%%%%%

\usepackage{amssymb,amsmath,amsthm,amscd,mathrsfs,graphicx, color}
\usepackage[cmtip,all,matrix,arrow,tips,curve]{xy}
\usepackage[notref,notcite]{showkeys}
%\usepackage[colorlinks]{hyperref}
\usepackage{multicol}
\usepackage{hyperref}
\usepackage[usenames,dvipsnames]{xcolor}
\hypersetup{colorlinks=true,citecolor=OliveGreen,linkcolor=BrickRed,urlcolor=BlueViolet}
\usepackage[active]{srcltx}
\usepackage{mathpazo}
\usepackage{setspace}\doublespacing 
%Double space, and make it easier for the grader to grade your homework.
%\usepackage{fullpage} 
%Use wide margins, and make it easier for the grader to grade your homework.


%%%%%%%%%%%%%%%%%%%%%%%%%%%%%%%%%%%%%%%%%%%%%% TIKZ FOR GRAPHING %%%%%%%%%%%%%%%%%%%%%%%%%%%%%%%%%%%%%%%%%%%%%%%%%%%%%%%
\usepackage{tikz,pgfplots}


%%%%%%%%%%%%%%%%%%%%%%%%%%%%%%%%%%%%%%%%%%%%%% 		ANSWER BOXES 		%%%%%%%%%%%%%%%%%%%%%%%%%%%%%%%%%%%%%%%%%%%
 \setlength\fboxsep{.3cm}
\setlength\fboxrule{.05cm}

\newcommand*{\boxedcolor}{red}
\makeatletter
\renewcommand{\boxed}[1]{\textcolor{\boxedcolor}{%
  \fbox{\normalcolor\m@th$\displaystyle#1$}}}
\makeatother

\makeatletter
\newcommand{\boxedred}[1]{\textcolor{red}{%
  \fbox{\normalcolor\m@th$\displaystyle#1$}}}
\makeatother

\makeatletter
\newcommand{\boxedblue}[1]{\textcolor{blue}{%
  \fbox{\normalcolor\m@th$\displaystyle#1$}}}
\makeatother




%%%%%%%%%%%%%%%%%%%%%%%%%%%%%%%%%%%%%%%%%%%%%%%%%%%% THEOREM ENVIRONMENTS %%%%%%%%%%%%%%%%%%%%%%%%%%%%%%%%


\numberwithin{equation}{section}
\newtheorem{teo}{Theorem}[section]
\newtheorem{pro}[teo]{Proposition}
\newtheorem{lem}[teo]{Lemma}
\newtheorem{cor}[teo]{Corollary}
\newtheorem{con}[teo]{Conjecture}
\newtheorem{convention}[teo]{}



\newtheorem{teoalpha}{Theorem}
\renewcommand{\theteoalpha}{\Alph{teoalpha}}
\newtheorem{proalpha}[teoalpha]{Proposition}
\newtheorem{coralpha}[teoalpha]{Corollary}


\theoremstyle{definition}
\newtheorem{dfn}[teo]{Definition}
\newtheorem{exa}[teo]{Example}
\newtheorem{que}[teo]{Question}

\theoremstyle{remark}
\newtheorem{rem}[teo]{Remark}
\newtheorem{nte}[teo]{Note}

%%%%%%%%%%%%%%%%%%%%%%%%%%%%%%%%%%%%%%%%%%%%%%% SOME CONDITIONALS FOR NOTES %%%%%%%%%%%%%%%%%%%%%%%%%%%%%%%%%%%

% Declare a new conditional
\newif\ifnotes
\notestrue 	% Show details
%\notesfalse 	% Exclude details


%%%%%%%%%%%%%%%%%%%%%%%%%%%%%%%%%%%%%%%%%%%%% COMMENTS %%%%%%%%%%%%%%%%%%%%%%%%%%%%

\newcommand{\marg}[1]{\normalsize{{\color{red}\footnote{{\color{blue}#1}}}{\marginpar[\vskip -.3cm {\color{BrickRed}\hfill\thefootnote$\implies$}]{\vskip -.3cm{ \color{BrickRed}$\impliedby$\thefootnote}}}}}

\newcommand{\qc}[1]{\marg{#1}}

%%%%%%%%%%%%%%%%%%%%%%%%%%%%%%%%%%%%%%%%%%%%%%%%%%% BEGIN DOCUMENT %%%%%%%%%%%%%%%%%%%%%%%%%%%%%%%%%%%
 
\begin{document}

\bibliographystyle{amsalpha}


%%%%%%%%%%%%%%%%%%%%%%%%%%%%%%%%%%%%%%%%%%%%%%%% AUTHOR INFO %%%%%%%%%%%%%%%%%%%%%%%%%%%%%%%%%%%% 

\author[Casalaina]{Sebastian Casalaina}
\address{University of Colorado, Department of Mathematics,  Campus Box 395,
Boulder, CO 80309-0395}
\email{casa@math.colorado.edu}
\date{\today}
%\thanks{I would like to take this opportunity to thank my class for their support.}


%%%%%%%%%%%%%%%%%%%%%%%%%%%%%%%%%%%%%%%%%%%%%%%% TITLE AND ABSTRACT %%%%%%%%%%%%%%%%%%%%%%%%%%%%%%%%%%%%

\title[Homework 1]{Homework 1 \\ \ \\  MATH 2001}

\begin{abstract} 
This is the first homework assignment.  The problems are from Hammack \cite[Ch.~1, \S 1.1]{H13}:
\begin{itemize}

\item \textbf{Chapter 2}  
\textbf{Section 2.9}, Exercises:  2, 3, 4, 5.
\textbf{Section 2.10}, Exercises:  2, 4, 10.
\textbf{Section 4}, Exercises:  1, 2, 5, 7, 9.

\end{itemize}
\end{abstract}


\maketitle


\tableofcontents

%%%%%%%%%%%%%%%%%%%%%%%%%%%%%%%%%%%%%%%%%%%%%%%% HOMEWORK ASSIGNMENT %%%%%%%%%%%%%%%%%%%%%%%%%%%%%%%%%%%%

%%%%%%%%%%%%%%%%%%%%%%%%%%%%%%%%%%%%%%%%%%%%%%%%%%%%%%%%%%%%%%%%%%%%%%%%%%%%%%%%%%%%%%%%%% CHAPTER 1 %%%%%%%%%%%%%%%%%%%%%%%%%%%%%%%%%%%%%%%%%%%%%%%%%%%%%%%%%%%%%%%%%%%%%%%%%%%%%%%%%%%%%%%%%%%%%%%%%%%


%%%%%%%%%%%%%%%%%%%%%%%%%%%%%%%%%%%%%%%%%%%%%%%% SECTION 1.1 %%%%%%%%%%%%%%%%%%%%%%%%%%%%%%%%%%%%

\section*{Chapter 2}


%%%%%%%%%%%%%%%%%%%%%%%%%%%%%%%%%%%%%%%%%%%%%%%% EXERCISE 2 %%%%%%%%%%%%%%%%%%%%%%%%%%%%%%%%%%%%

\subsection*{Ch.2, \S 2.9,  Exercise 2, 3, 4, 5}
Translate each of the following sentences into symbolic logic.

\begin{enumerate}

\item[2.]
The number x is positive but the number y is not positive.

\item[3.]
If x is prime, then $ \sqrt{x} $ is not a rational number.

\item[4.]
For every prime number p there is another prime number q with q > p.

\item[5.]
For every positive number $ \varepsilon $, there is a positive number $ \delta $ for which $ |x - a| < \delta $ implies $ |f(x)-f(a)| < \varepsilon $.

\end{enumerate}


\begin{proof}[Solution to Ch.2, \S 2.9,  Exercise 2, 3, 4, 5] \ \\

\begin{center}

\item[2.]
(x is positive) $ \land $ (y is not positive)

\item[3.]
(x is prime) $ \implies $ ($ \sqrt{x} $ is not prime)

\item[4.]
$ \exists q \in \textbf{primes} $, ($ \forall p \in \textbf{primes}, q > p $)

\item[5.]
$ \exists \delta, (\delta > 0) \land [\forall \varepsilon, ((\varepsilon > 0) \land (|x-\alpha| < \delta)) \implies (f(x)-f(a)| < \varepsilon] $

\end{center}

\end{proof}


%%%%%%%%%%%%%%%%%%%%%%%%%%%%%%%%%%%%%%%%%%%%%%%% EXERCISE 8 %%%%%%%%%%%%%%%%%%%%%%%%%%%%%%%%%%%%


\subsection*{Ch.2, \S 2.10,  Exercise 2, 4, 10}
Negate the following sentences.

\begin{enumerate}

\item[2.]
If x is prime, then $ \sqrt{x} $ is not a rational number.

\item[4.]
For every positive number $ \varepsilon $, there is a positive number $ \delta $ such that $ |x-a| < \delta $ implies $ |f(x)-f(a)| < \varepsilon $.

\item[10.]
If $ f $ is a polynomial and its degree is greeter than 2, then $ f' $ is not constant.

\end{enumerate}


\begin{proof}[Solution to Ch.2, \S 2.10,  Exercise 2, 4, 10] \ \\

\begin{center}

\begin{enumerate}

\item[2.]
x is prime, $ \sqrt{x} $ is a rational number.

\item[4.]
There is a positive number $ \varepsilon $, such that for all positive number $ \delta $, $ (|x-a| < \delta) \land (|f(x)-f(a)| \geq \varepsilon) $.

\item[10.]
$ f $ is a polynomial and it s degree is greeter than 2, $ f' $ is constant.


\end{enumerate}

\end{center}

\end{proof}


%%%%%%%%%%%%%%%%%%%%%%%%%%%%%%%%%%%%%%%%%%%%%%%% EXERCISE 18 %%%%%%%%%%%%%%%%%%%%%%%%%%%%%%%%%%%%


\subsection*{Ch.4, \S 4,  Exercise 1}  Use the method of direct proof to prove the following statements: If x is an even integer, then $ x^2 $ is even.

\begin{proof}[Solution to Ch.4, \S 4,  Exercise 1]  \ \\
\textbf{Proposition}  If x is an even integer, then $ x^2 $ is even.\\
\textit{Proof}. Suppose x is even. Then $ x = 2a $ for some $ a \in \mathbb{Z} $, by definition of an even number. Thus $ x^2 = 4a^2 = 2(2a^2) $, so $ x^2 = 2b $ where $ b = 2a^2 \in \mathbb{Z} $. Therefore $ x^2 $ is even, by definition of an even number.


\end{proof}


%%%%%%%%%%%%%%%%%%%%%%%%%%%%%%%%%%%%%%%%%%%%%%%% EXERCISE 30 %%%%%%%%%%%%%%%%%%%%%%%%%%%%%%%%%%%%


\subsection*{Ch.4, \S 4,  Exercise 3}  Use the method of direct proof to prove the following statements: If $ a $ is an odd integer, then $ a^2 + 3a + 5 $ is odd.

\begin{proof}[Solution to Ch.1, \S 1.1,  Exercise 30] \ \\
\textbf{Proposition} If $ a $ is an odd integer, then $ a^2 + 3a + 5 $ is odd. \ \\
\textit{Proof}. Suppose $ a $ is odd. Then $ a = 2b + 1 $ for some $ b \in \mathbb{Z} $, by definition of an odd number. Thus $ a^2 + 3a + 5 = 4b^2 + 10b + 9 = 2(2b^2 + 5b + 4) + 1 $, so $ a^2 + 3a + 5 = 2b + 1 $ where $ b = (2b^2 + 5b + 4) \in \mathbb{Z} $. Therefor $ a^2 + 3a + 5 $ is odd, by definition of an odd number.
\end{proof}



%%%%%%%%%%%%%%%%%%%%%%%%%%%%%%%%%%%%%%%%%%%%%%%% EXERCISE 38 %%%%%%%%%%%%%%%%%%%%%%%%%%%%%%%%%%%%


\subsection*{Ch.4, \S 4,  Exercise 5}  Use the method of direct proof to prove the following statements: Suppose $ x,y \in \mathbb{Z} $. If $ x $ is even, then $ xy $ is even.


\begin{proof}[Solution to Ch.1, \S 1.1,  Exercise 38]\ \\
\textbf{Proposition} Suppose $ x,y \in \mathbb{Z} $. If $ x $ is even, then $ xy $ is even.\ \\
\textit{Proof}. x is even. Then $ x = 2a $ for some $ a \in \mathbb{Z} $, by definition of even numbers.\\
\textbf{Case 1}. Suppose y is odd. Then $ y = 2b + 1 $ for some $ b \in \mathbb{Z} $, by definition of odd numbers. Thus $ x*y = 2a * (2b + 1) = 4ab + 2a = 2(2ab + a) $, so $ x*y = 2c $ where $ c = (2ab + a) \in \mathbb{Z} $. Therefor $ x*y $ is even, by definition of an even number.\\
\textbf{Case 2}. Suppose y is even. Then $ y = 2d $ for some $ d \in \mathbb{Z} $, by definition of an even number. Thus $ x * y = 2a * 2d = 4ad = 2(2ad) $, so $ x * y = 2e $ where $ e = 2ad \in \mathbb{Z} $. Therefor $ x*y $ is even, by definition of an even number.\\
Because $ y \in \mathbb{Z} $, y is either even or odd. In both cases, $ x*y $ is even.

\end{proof}


%%%%%%%%%%%%%%%%%%%%%%%%%%%%%%%%%%%%%%%%%%%%%%%% EXERCISE 40 %%%%%%%%%%%%%%%%%%%%%%%%%%%%%%%%%%%%


\subsection*{Ch.4, \S 4,  Exercise 7} Use the method of direct proof to prove the following statements: Suppose $ a,b \in \mathbb{Z} $. If $ a | b $, then $ a^2 | b^2 $.


\begin{proof}[Solution to Ch.1, \S 1.1,  Exercise 40] \ \\
\textbf{Proposition} Suppose $ a,b \in \mathbb{Z} $. If $ a | b $, then $ a^2 | b^2 $.\ \\
\textit{Proof} Suppose $ a | b $ where $ a, b \in \mathbb{Z} $. Then $ a = b * c $ for some $ c \in \mathbb{Z} $, by definition. Thus $ a^2 = (b * c)^2 = b^2 * c^2 $, so $ a^2 = b^2 * d $ where $ d = c^2 \in \mathbb{Z} $. There for $ a^2|b^2 $, by definition.


\end{proof}



\newpage

\ifnotes


\else
	This is not the full version.  This can be useful if there is scratch work you want to keep for yourself, but you do not want other people to see. 
\fi




\bibliography{templateHW}
\end{document}
